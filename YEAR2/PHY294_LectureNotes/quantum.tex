\section{The Schrödinger Equation}

\subsection{Review of Wavefunctions and Operators}
\begin{definition}[The Schrödinger Equation] The time-dependent Schrödinger equation for a single non-relativistic particle in one dimension is given by
\begin{equation}
    i\hbar \frac{\partial}{\partial t} \Psi(x,t) = -\frac{\hbar^2}{2m} \frac{\partial^2}{\partial x^2} \Psi(x,t) + V(x) \Psi(x,t)
\end{equation}
where $\Psi(x,t)$ is the wavefunction of the particle, $V(x)$ is the potential energy, $m$ is the mass of the particle, and $\hbar$ is the reduced Planck's constant.

We denote the the momentum operator as
\begin{subequations}
\begin{equation}
    \hat{p} = -i\hbar \frac{\partial}{\partial x}
\end{equation}
and the Hamiltonian operator as the sum of kinetic and potential energy operators:
\begin{equation}
    \hat{H} = \frac{\hat{p}^2}{2m} + V(x) = -\frac{\hbar^2}{2m} \frac{\partial^2}{\partial x^2} + V(x)
\end{equation}
and the energy operator as
\begin{equation}
    \hat{E} = i\hbar \frac{\partial}{\partial t}
\end{equation}
\end{subequations}



\end{definition}


\begin{definition}[Expectation Value] The expectation value of an operator $\hat{O}$ in a state described by the wavefunction $\Psi(x,t)$ is given by
\begin{equation}
    \langle \hat{O} \rangle = \int_{-\infty}^{\infty} \Psi^*(x,t) \hat{O} \Psi(x,t) \, dx
\end{equation}
where $\Psi^*(x,t)$ is the complex conjugate of the wavefunction.
\end{definition}

\begin{example}
    The expectation value of the momentum operator $\hat{p}$ is given by
    \begin{equation}
        \langle \hat{p} \rangle = \int_{-\infty}^{\infty} \Psi^*(x,t) \left(-i\hbar \frac{\partial}{\partial x}\right) \Psi(x,t) \, dx
    \end{equation}

    We also write this as:
    \begin{equation}
        \langle \hat{p} \rangle = -i\hbar  \langle \Psi | \frac{\partial}{\partial x} | \Psi \rangle
    \end{equation}
\end{example}

\begin{definition}[Infinite Square Well Potential] The infinite square well potential is defined as
\begin{equation}
    V(x) = \begin{cases}
    0 & \text{for } 0 < x < a \\
    \infty & \text{otherwise}
    \end{cases}
\end{equation}
where $a$ is the width of the well.

The normalized stationary state wavefunctions for the infinite square well are given by
\begin{equation}
    \psi_n(x) = \sqrt{\frac{2}{a}} \sin\left(\frac{n\pi x}{a}\right) \quad \text{for }n = 1, 2, 3, \ldots
\end{equation}
with corresponding energy eigenvalues
\begin{equation}
    E_n = \frac{n^2 \pi^2 \hbar^2}{2ma^2}
\end{equation}
Thus, the momentum eigenvalues are
\begin{equation}
    p_n = \pm \frac{n\pi \hbar}{a}
\end{equation}

Then, according to uncertainty principle, the uncertainty in position $\Delta x$ and uncertainty in momentum $\Delta p$ satisfy
\begin{equation}
    \Delta x \Delta p \geq \frac{\hbar}{2}
\end{equation}
\end{definition}