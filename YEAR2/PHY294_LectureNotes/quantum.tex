\section{The Schrödinger Equation}

\begin{keybox}
\textbf{Key Equations:}
\begin{itemize}
    \item Time-dependent Schrödinger equation: $i\hbar \frac{\partial}{\partial t} \Psi(x,t) = \hat{H} \Psi(x,t)$
    \item Momentum operator: $\hat{p} = -i\hbar \frac{\partial}{\partial x}$
    \item Hamiltonian: $\hat{H} = \frac{\hat{p}^2}{2m} + V(x)$
    \item Energy operator: $\hat{E} = i\hbar \frac{\partial}{\partial t}$
\end{itemize}
\textbf{Tip:} The Schrödinger equation governs the time evolution of quantum systems.
\end{keybox}

Quantum mechanics describes the behavior of particles at the atomic and subatomic scales. The fundamental equation governing this behavior is the Schrödinger equation, which replaces classical mechanics for quantum systems.

\subsection{Time-Dependent Schrödinger Equation}

\begin{definition}[The Schrödinger Equation] The time-dependent Schrödinger equation for a single non-relativistic particle in one dimension is given by
\begin{equation}
    i\hbar \frac{\partial}{\partial t} \Psi(x,t) = -\frac{\hbar^2}{2m} \frac{\partial^2}{\partial x^2} \Psi(x,t) + V(x) \Psi(x,t)
\end{equation}
where $\Psi(x,t)$ is the wavefunction of the particle, $V(x)$ is the potential energy, $m$ is the mass of the particle, and $\hbar$ is the reduced Planck's constant.
\end{definition}

We can rewrite this equation using operators. The momentum operator is defined as
\begin{equation}
    \hat{p} = -i\hbar \frac{\partial}{\partial x}
\end{equation}
and the Hamiltonian operator, representing the total energy, is
\begin{equation}
    \hat{H} = \frac{\hat{p}^2}{2m} + V(x) = -\frac{\hbar^2}{2m} \frac{\partial^2}{\partial x^2} + V(x)
\end{equation}
The energy operator is
\begin{equation}
    \hat{E} = i\hbar \frac{\partial}{\partial t}
\end{equation}
Thus, the Schrödinger equation becomes
\begin{equation}
    \hat{H} \Psi = i\hbar \frac{\partial}{\partial t} \Psi
\end{equation}

\subsection{Properties of Wavefunctions}

For the wavefunction to represent a physical state, it must satisfy certain properties.

\begin{definition}[Properties of Wavefunctions] The wavefunction $\Psi(x,t)$ must satisfy the following properties:
\begin{itemize}
    \item \textbf{Normalization:} The total probability of finding the particle somewhere in space must be 1:
    \begin{equation*}
        \int_{-\infty}^{\infty} |\Psi(x,t)|^2 \, dx = 1
    \end{equation*}
    \item \textbf{Continuity:} The wavefunction and its first derivative must be continuous everywhere.
    \item \textbf{Second-order discontinuities:} The second derivative of the wavefunction may be discontinuous at points where the potential $V(x)$ has infinite discontinuities (e.g., infinite potential walls). The discontinuity depends on the sign of $V(x) - E$ at the point of discontinuity.
    \item \textbf{Single-valuedness:} The wavefunction must have a unique value at each point in space and time.
    \item \textbf{Decay at infinity:} The wavefunction must approach zero as $x$ approaches $\pm \infty$ to ensure normalizability.
\end{itemize}
\end{definition}

\subsection{Expectation Values}

In quantum mechanics, we often deal with expectation values rather than definite values.

\begin{definition}[Expectation Value] The expectation value of an operator $\hat{O}$ in a state described by the wavefunction $\Psi(x,t)$ is given by
\begin{equation}
    \langle \hat{O} \rangle = \int_{-\infty}^{\infty} \Psi^*(x,t) \hat{O} \Psi(x,t) \, dx
\end{equation}
where $\Psi^*(x,t)$ is the complex conjugate of the wavefunction.
\end{definition}

\begin{example}
    The expectation value of the momentum operator $\hat{p}$ is given by
    \begin{equation*}
        \langle \hat{p} \rangle = \int_{-\infty}^{\infty} \Psi^*(x,t) \left(-i\hbar \frac{\partial}{\partial x}\right) \Psi(x,t) \, dx
    \end{equation*}

    We also write this as:
    \begin{equation*}
        \langle \hat{p} \rangle = -i\hbar  \langle \Psi | \frac{\partial}{\partial x} | \Psi \rangle
    \end{equation*}
\end{example}

\section{Quantum Systems in One Dimension}

\subsection{Infinite Square Well}

\begin{keybox}
\textbf{Key Equations for Infinite Square Well:}
\begin{itemize}
    \item Potential: $V(x) = 0$ for $0 < x < a$, $\infty$ otherwise
    \item Wavefunctions: $\psi_n(x) = \sqrt{\frac{2}{a}} \sin\left(\frac{n\pi x}{a}\right)$
    \item Energies: $E_n = \frac{n^2 \pi^2 \hbar^2}{2ma^2}$
    \item Momentum: $p_n = \pm \frac{n\pi \hbar}{a}$
\end{itemize}
\textbf{Tip:} Remember the boundary conditions: $\psi(0) = \psi(a) = 0$.
\end{keybox}

The infinite square well is one of the simplest quantum systems and illustrates quantization of energy levels.

\begin{definition}[Infinite Square Well Potential] The infinite square well potential is defined as
\begin{equation*}
    V(x) = \begin{cases}
    0 & \text{for } 0 < x < a \\
    \infty & \text{otherwise}
    \end{cases}
\end{equation*}
where $a$ is the width of the well.
\end{definition}

The normalized stationary state wavefunctions for the infinite square well are given by
\begin{equation}
    \psi_n(x) = \sqrt{\frac{2}{a}} \sin\left(\frac{n\pi x}{a}\right) \quad \text{for }n = 1, 2, 3, \ldots
\end{equation}
with corresponding energy eigenvalues
\begin{equation}
    E_n = \frac{n^2 \pi^2 \hbar^2}{2ma^2}
\end{equation}

The momentum eigenvalues are
\begin{equation*}
    p_n = \pm \frac{n\pi \hbar}{a}
\end{equation*}

According to the uncertainty principle, the uncertainty in position $\Delta x$ and uncertainty in momentum $\Delta p$ satisfy
\begin{equation*}
    \Delta x \Delta p \geq \frac{\hbar}{2}
\end{equation*}

\subsection{Finite Square Well}

\begin{keybox}
\textbf{Key Equations for Finite Square Well:}
\begin{itemize}
    \item Potential: $V(x) = 0$ for $0 < x < a$, $V_0$ otherwise
    \item Bound states: $E < V_0$, oscillatory inside, exponential decay outside
    \item Scattering states: $E > V_0$, traveling waves everywhere
\end{itemize}
\textbf{Tip:} Bound states have discrete energies, scattering states have continuous spectrum.
\end{keybox}

The finite square well allows particles to tunnel through the barriers, unlike the infinite well.

\begin{definition}[Finite Square Well Potential]
The finite square well potential is defined as
\begin{equation*}
V(x) =
\begin{cases}
0 & 0 < x < a \\
V_0 & \text{otherwise}
\end{cases}
\end{equation*}
\end{definition}

The time-independent Schrödinger equation is
\begin{equation}
-\frac{\hbar^2}{2m}\frac{d^2\psi}{dx^2} + V(x)\psi = E\psi
\end{equation}
which can be written as a curvature condition
\begin{equation*}
\frac{d^2\psi}{dx^2} = \frac{2m}{\hbar^2}(V(x)-E)\psi
\end{equation*}

The energy is related to the wavevector and wavelength by
\begin{equation*}
E = \frac{\hbar^2 k^2}{2m}, \quad k = \frac{2\pi}{\lambda}, \quad E \propto \frac{1}{\lambda^2}
\end{equation*}

\subsubsection{Bound States ($E < V_0$)}

For bound states, the wavefunction is oscillatory inside the well ($E > V(x)$) and decays exponentially outside ($E < V(x)$):
\begin{equation*}
\psi(x) =
\begin{cases}
C\, e^{\kappa x} & x < 0 \\[6pt]
A \sin(kx) + B \cos(kx) & 0 < x < a \\[6pt]
D\, e^{-\kappa (x - a)} & x > a
\end{cases}
\end{equation*}
where
\begin{equation*}
k = \sqrt{\frac{2mE}{\hbar^2}}, \quad \kappa = \sqrt{\frac{2m(V_0 - E)}{\hbar^2}}
\end{equation*}

Because the wavefunction penetrates the finite walls, the effective wavelength inside the well is larger than that of an infinite square well of the same width. Consequently,
\begin{equation*}
\lambda_n^{(\text{finite})} > \lambda_n^{(\infty)} \quad \Longrightarrow \quad E_n^{(\text{finite})} < E_n^{(\infty)}
\end{equation*}

Only discrete energies allow the curvature of the oscillatory solution inside the well to match smoothly onto the exponential decay outside, producing quantized bound states.

\subsubsection{Scattering States ($E > V_0$)}

For scattering states, the particle propagates in all regions with momentum
\begin{equation*}
p_x = \pm \hbar k, \quad k = \sqrt{\frac{2mE}{\hbar^2}}
\end{equation*}

The wavefunction is
\begin{equation*}
\psi(x) =
\begin{cases}
C\, e^{iqx} + D\, e^{-iqx} & x < 0 \\[6pt]
A\, e^{ikx} + B\, e^{-ikx} & 0 < x < a \\[6pt]
F\, e^{iq(x - a)} + G\, e^{-iq(x - a)} & x > a
\end{cases}
\end{equation*}
where
\begin{equation*}
q = \sqrt{\frac{2m(E - V_0)}{\hbar^2}}
\end{equation*}

In this case, traveling-wave solutions exist in all regions and the energy spectrum is continuous.

\section{Quantum Harmonic Oscillator}

\begin{keybox}
\textbf{Key Equations for Quantum Harmonic Oscillator:}
\begin{itemize}
    \item Potential: $V(x) = \frac{1}{2} k x^2 = \frac{1}{2} m \omega^2 x^2$ where $k$ is the spring constant
    \item Angular frequency: $\omega = \sqrt{k/m}$
    \item Energy levels: $E_n = \hbar \omega (n + \frac{1}{2})$ for $n = 0,1,2,\dots$
    \item Wavefunctions: $\psi_n(x) = \frac{1}{\sqrt{2^n n!}} \left(\frac{m\omega}{\pi\hbar}\right)^{1/4} H_n(\xi) e^{-\xi^2/2}$ where $\xi = \sqrt{\frac{m\omega}{\hbar}} x$
\end{itemize}
\textbf{Tip:} The zero-point energy $E_0 = \frac{1}{2} \hbar \omega$ is a purely quantum effect.
\end{keybox}

Many physical systems can be approximated as harmonic oscillators near their equilibrium points, such as atoms in molecules or lattice vibrations in solids.

The potential energy for a harmonic oscillator is
\begin{equation}
V(x) = \frac{1}{2} k x^2 = \frac{1}{2} m \omega^2 x^2
\end{equation}
where the angular frequency is given by
\begin{equation}
\omega = \sqrt{\frac{k}{m}}
\end{equation}
and $k$ is the spring constant (force constant) that determines the stiffness of the oscillator.

The time-independent Schrödinger equation becomes
\begin{equation*}
-\frac{\hbar^2}{2m} \frac{d^2 \psi}{dx^2} + \frac{1}{2} m \omega^2 x^2 \psi = E \psi
\end{equation*}

The solutions involve Hermite polynomials, and the energy levels are quantized as
\begin{equation}
E_n = \hbar \omega \left(n + \frac{1}{2}\right)
\end{equation}
where $n = 0, 1, 2, \ldots$

The wavefunctions are given by
\begin{equation*}
\psi_n(x) = \frac{1}{\sqrt{2^n n!}} \left(\frac{m\omega}{\pi\hbar}\right)^{1/4} H_n\left(\sqrt{\frac{m\omega}{\hbar}} x\right) \exp\left(-\frac{m\omega x^2}{2\hbar}\right)
\end{equation*}
where $H_n(\xi)$ are the Hermite polynomials.

\begin{definition}[Hermite Polynomials]
The Hermite polynomials $H_n(\xi)$ are defined by Rodrigues' formula:
\begin{equation*}
H_n(\xi) = (-1)^n e^{\xi^2} \frac{d^n}{dx^n} e^{-\xi^2}
\end{equation*}
for $n = 0, 1, 2, \ldots$ where $\xi = \sqrt{\frac{m\omega}{\hbar}} x$.

The first few Hermite polynomials are:
\begin{align*}
H_0(\xi) & = 1 \\
H_1(\xi) & = 2\xi \\
H_2(\xi) & = 4\xi^2 - 2 \\
H_3(\xi) & = 8\xi^3 - 12\xi \\
H_4(\xi) & = 16\xi^4 - 48\xi^2 + 12
\end{align*}

They satisfy the orthogonality condition:
\begin{equation*}
\int_{-\infty}^{\infty} H_m(\xi) H_n(\xi) e^{-\xi^2} d\xi = \sqrt{\pi} 2^n n! \delta_{mn}
\end{equation*}
\end{definition}

\begin{definition}[Properties of the Quantum Harmonic Oscillator]
The quantum harmonic oscillator has the following properties:
\begin{itemize}
    \item \textbf{Orthogonality:} The wavefunctions $\psi_n(x)$ are orthogonal:
    \begin{equation*}
        \int_{-\infty}^{\infty} \psi_m^*(x) \psi_n(x) \, dx = \delta_{mn}
    \end{equation*}
    \item \textbf{Zero-point Energy:} The ground state ($n=0$) has energy $\frac{1}{2} \hbar \omega$, which cannot be zero due to the uncertainty principle.
    \item \textbf{Equally Spaced Levels:} Energy levels are equally spaced by $\hbar \omega$.
    \item \textbf{Probability Distribution:} $|\psi_n(x)|^2$ gives the probability density for finding the particle at position $x$ in state $n$.
\end{itemize}
\end{definition}

\section{Time-Dependent Schrödinger Equation}
The time-dependent Schrödinger equation describes how quantum states evolve over time.
\begin{definition}[Time-Dependent Schrödinger Equation]
The time-dependent Schrödinger equation is given by
\begin{equation}
i\hbar \frac{\partial}{\partial t} \Psi(x,t) = \hat{H} \Psi(x,t)
\end{equation}
where $\hat{H}$ is the Hamiltonian operator.
\end{definition} 

\begin{definition}[Free Particle Solution for TDSE]
For a free particle (i.e., $V(x) = 0$), the time-dependent Schrödinger equation simplifies to
\begin{equation}
i\hbar \frac{\partial}{\partial t} \Psi(x,t) = -\frac{\hbar^2}{2m} \frac{\partial^2}{\partial x^2} \Psi(x,t)
\end{equation}
A general solution can be expressed as a superposition of plane waves:
\begin{equation}
\Psi(x,t) = \int_{-\infty}^{\infty} A(k) e^{i(kx - \omega t)} \, dk, \quad \Psi_k(x,t) = e^{i(kx - \omega t)}
\end{equation}
where $\omega = \frac{\hbar k^2}{2m}$ and $A(k)$ is the amplitude distribution in momentum space.
\end{definition}

\begin{keybox}
\textbf{Separation of Variables for TDSE:} To solve the time-dependent Schrödinger equation, we can use the method of separation of variables by assuming a solution of the form $\Psi(x,t) = \psi(x) T(t)$. This leads to two separate equations:
\begin{itemize}
    \item Spatial part (time-independent Schrödinger equation):
    \begin{equation*}
    \hat{H} \psi(x) = E \psi(x)
    \end{equation*}
    \item Temporal part:
    \begin{equation*}
    i\hbar \frac{dT(t)}{dt} = E T(t)
    \end{equation*}
\end{itemize}
The temporal equation has the solution:
\begin{equation*}
T(t) = e^{-iEt/\hbar}
\end{equation*}
Thus, the full solution is:
\begin{equation}
\Psi(x,t) = \psi(x) e^{-iEt/\hbar}
\end{equation}
\end{keybox}

\begin{example}[Simplest Non-trivial Solution to TDSE: infinite square well]
Consider a particle in an infinite square well of width \(a\). The time-independent Schrödinger equation gives the stationary states:
\begin{align*}
\psi_n(x) &= \sqrt{\frac{2}{a}} \sin\left(\frac{n\pi x}{a}\right), \\
E_n &= \frac{n^2 \pi^2 \hbar^2}{2ma^2}
\end{align*}
Using separation of variables, the time-dependent solution is:
\begin{equation}
\Psi_n(x,t) = \psi_n(x) e^{-iE_n t/\hbar} = \sqrt{\frac{2}{a}} \sin\left(\frac{n\pi x}{a}\right) e^{-i \frac{n^2 \pi^2 \hbar}{2ma^2} t}
\end{equation}
This describes the quantum state of the particle in the infinite square well over time.
\end{example}

\begin{definition}[Time-Dependent Probability in Superposition States]
For a superposition of two energy eigenstates, the total wavefunction is
\begin{equation}
\Psi(x,t) = a_1 \Psi_1(x,t) + a_2 \Psi_2(x,t),
\end{equation}
where the coefficients satisfy the normalization condition
\begin{equation}
|a_1|^2 + |a_2|^2 = 1.
\end{equation}

Each energy eigenstate has time dependence
\begin{equation}
\Psi_n(x,t) = \psi_n(x)e^{-iE_n t/\hbar}.
\end{equation}
Substituting,
\begin{equation}
\Psi(x,t)
= a_1 \psi_1(x)e^{-iE_1 t/\hbar}
+ a_2 \psi_2(x)e^{-iE_2 t/\hbar}.
\end{equation}

The time-dependent probability density is obtained by taking the modulus squared:
\begin{equation}
|\Psi(x,t)|^2
= \left| a_1 \psi_1(x) + a_2 \psi_2(x)e^{-i(E_2-E_1)t/\hbar} \right|^2.
\end{equation}

Using the identity $|U + V e^{i\theta}|^2 = U^2 + V^2 + 2UV\cos\theta$, the probability density becomes
\begin{equation}
|\Psi(x,t)|^2
= \underbrace{|a_1|^2|\psi_1(x)|^2 + |a_2|^2|\psi_2(x)|^2}_{\text{time-independent}}
+ \underbrace{2a_1a_2\psi_1(x)\psi_2(x) \cos\!\left(\frac{(E_2-E_1)t}{\hbar}\right)}_{\text{time-dependent}}.
\end{equation}

This shows that interference between energy eigenstates produces a time-dependent oscillation in the probability density with angular frequency $(E_2-E_1)/\hbar$.
\end{definition}
