\documentclass[11pt]{report}
\usepackage{mathtools}
\usepackage{amsmath}
\usepackage{amssymb}
\usepackage{amsfonts}
\usepackage{amsthm}
\usepackage{xcolor}
\usepackage{graphicx}
\usepackage[top=2.0cm,bottom=2.0cm,left=2.5cm,right=2.5cm]{geometry}
\usepackage{tikz}
\usepackage{float}
\usepackage{multicol}
\usepackage{pgfplots}
\usepackage{lastpage}
\usepackage{siunitx}
\usepackage{xspace}
\usepackage[labelfont=bf]{caption}
\usepackage[hidelinks, urlcolor=blue, linkcolor=blue, colorlinks=true]{hyperref}
\usepackage[capitalize,noabbrev]{cleveref}
\usepackage[absolute]{textpos}
\usepackage{systeme}

\newcommand{\R}{\mathbb{R}}
\newcommand{\C}{\mathbb{C}}
%% define course title
\newcommand{\course}{MAT185}
\newcommand{\assignmenttitle}{Assignment 1}

%% header and footer
\firstpageheader{}{}{\textbf{{\color{red} Due:} 10:00pm, Tuesday Jan. 21, 2025}}
\firstpageheadrule
\runningheader{}{Page~\thepage~of~\numpages}{\course~--~\assignmenttitle}
\footer{}{}{}

\setlength\parindent{0pt} % no indentation in document

%% formats exam class
\qformat{\textbf{Question \thequestiontitle:}\hfill} % title of question 
\boxedpoints
\pointpoints{mark}{marks}
\pointsinrightmargin
\hpword{Marks:}
\hsword{Your score:}
\unframedsolutions
\totalformat{\boxed{\textnormal{\totalpoints~\if\totalpoints1 mark\else marks\fi}}}
\definecolor{SolutionColor}{rgb}{0,0,1}
\renewcommand{\solutiontitle}{}
\AtBeginEnvironment{solution}{\color{blue}}

% %correct choices in solution
\CorrectChoiceEmphasis{\rm}
\checkedchar{\tikz\draw[blue,fill=blue] (0,0) circle (1ex);}

% % increase distance between checkbox items
\renewcommand{\checkboxeshook}{\setlength{\itemsep}{6pt}}

%% distance between questions and parts
\renewcommand{\questionshook}{\setlength{\parsep}{10pt}}
\renewcommand{\partshook}{\setlength{\parsep}{15pt}}

%% define arrows in text
\newcommand{\arrow}{$\rightarrow$\xspace}
\newcommand{\Arrow}{$\Rightarrow$\xspace}

% % math notation:
%\veccol{1}{2}{3}
\newcommand{\veccol}[3]{
    \begin{bmatrix}
        #1\\
        #2\\
        #3\\
    \end{bmatrix}}
  
%\vecrow{1}{2}{3}
\newcommand{\vecrow}[3]{\left[#1~#2~ #3\right]}

%\matrixTwo{1}{2}{3}{4}
\newcommand{\matrixTwo}[4]{\left[\begin{array}{cc}#1&#2\\#3&#4\end{array}\right]}

% \matrixThree{1}{2}{3}{4}{5}{6}{7}{8}{9}
\newcommand{\matrixThree}[9]{\left[\begin{array}{ccc}#1&#2&#3\\#4&#5&#6\\#7&#8&#9\end{array}\right]}

%\matrixCorner{1}{2}{3}{4}
\newcommand{\matrixCorner}[4]{\left[\begin{array}{ccc}#1& \cdots&#2\\ \vdots & \ddots & \vdots\\#3&
      \cdots&#4\end{array}\right]}

% \nR
\newcommand{\nR}{{}^{n}\mathbb{R}}
% \Rn
\newcommand{\Rn}{\mathbb{R}^{n}}
% \nRn
\newcommand{\nRn}{{}^{n}\mathbb{R}^{n}}
% \nRm
\newcommand{\nRm}{{}^{n}\mathbb{R}^{m}}
% \nRm
\newcommand{\mRn}{{}^{m}\mathbb{R}^{n}}
% \mRm
\newcommand{\mRm}{{}^{m}\mathbb{R}^{m}}        

% \u
\renewcommand{\u}{{\bf u}}      
% \v
\renewcommand{\v}{{\bf v}}      
% \w
\newcommand{\w}{{\bf w}}    
% \V
\newcommand{\V}{{\bf V}}                   
       
%% define abbreviations
\newcommand{\row}{\operatorname{row}\,}
\newcommand{\col}{\operatorname{col}\,}
\renewcommand{\dim}{\operatorname{dim}\,}
\renewcommand{\span}{\operatorname{span}\,}
\newcommand{\rank}{\operatorname{rank}\,}
\renewcommand{\ker}{\operatorname{ker}\,}
\newcommand{\nul}{\operatorname{null}\,}
\renewcommand{\det}{\operatorname{det}\,}
\newcommand{\adj}{\operatorname{adj}\,}

\usepackage{xcolor}
% Sean's original colours:
%\definecolor{dkrgreen}{rgb}{0.1, 0.4, 0.3} 
\definecolor{dkrgreen}{HTML}{009988} % this is the color-blind friendly teal from below
%\definecolor{dkred}{rgb}{0.8, 0.05, 0.05} 
\definecolor{dkred}{HTML}{EE3377}  % this is the colour-blind friendly magenta from below
%\definecolor{orange}{rgb}{0.8, 0.33, 0.0}
%\definecolor{goldenrod}{rgb}{0.85, 0.65, 0.13}
\definecolor{blue}{HTML}{1965B0} % this is the colour-blind friendly blue from below
%
% colour-blind-friendly colours from https://personal.sron.nl/~pault/
\definecolor{tolBlue}{HTML}{1965B0}
\definecolor{tolMedBlue}{HTML}{5289C7}
\definecolor{tolLightBlue}{HTML}{7BAFDE} 
\definecolor{tolRed}{HTML}{E8601C} 
\definecolor{tolYellow}{HTML}{F6C141}
\definecolor{tolTeal}{HTML}{009988}
%\definecolor{tolBlue}{HTML}{0077BB} 
\definecolor{tolCyan}{HTML}{33BBEE}
\definecolor{tolTeal}{HTML}{009988} 
\definecolor{tolOrange}{HTML}{EE7733} 
%\definecolor{tolRed}{HTML}{CC3311} 
\definecolor{tolMagenta}{HTML}{EE3377} 
\definecolor{tolGrey}{HTML}{BBBBBB}

%%% This command makes a framed box of a chosen height.
\newcommand{\makenonemptybox}[2]{%
\par\nobreak\vspace{\ht\strutbox}\noindent
\setlength{\fboxrule}{0pt} % set this to 0pt to make invisible
\fbox{%
\parbox[c][#1][t]{\dimexpr\linewidth-2\fboxsep}{
  \hrule width \hsize height 0pt
  \vspace{-0.6cm}
  \color{SolutionColor}#2\color{black}
 }%
}%
}


\begin{document}
\thispagestyle{empty}
{\LARGE \bf PHY 293 Lecture Notes}\\
{\large Hei Shing Cheung}\\
Waves and Modern Physics, Fall 2025 \hfill PHY293\\
\\
The up-to-date version of this document can be found at \url{https://github.com/HaysonC/skulenotes}\\

\chapter{Waves}
\section{Harmonic Oscillators}

\subsection{Governing Equations of Harmonic Oscillators}

\paragraph{Types of Harmonic Oscillators} There are three types of harmonic oscillators: simple, damped, and driven harmonic oscillators. Consider a simple one dimensional harmonic oscillator, they are defined by the following differential equations:

\begin{definition}[Simple Harmonic Oscillator]
    A simple harmonic oscillator is described by Hooke's law:

    \begin{equation} \label{eq:hooke}
        m \frac{d^2 x}{dt^2} + kx = 0
    \end{equation}
    where \( k \) is the spring constant, \( m \) is the mass, and \( x \) is the displacement from equilibrium.
\end{definition}

\begin{definition}[Damped Harmonic Oscillator]
    A damped harmonic oscillator is described by the following differential equation, by adding a damping term proportional to $\dot{x}$ to the simple harmonic oscillator equation:
    \begin{equation} \label{eq:damped_ho}
        m \frac{d^2 x}{dt^2} + b \frac{dx}{dt} + kx = 0   
    \end{equation}
    where \( b \) is the damping coefficient.
\end{definition}

\begin{definition}[Driven Harmonic Oscillator]
    A driven harmonic oscillator is described by the following differential equation, which includes an external driving force $F(t)$:
    \begin{equation}
        m \frac{d^2 x}{dt^2} + b \frac{dx}{dt} + kx = F(t)   
    \end{equation}
\end{definition}
\subsection{The Wave Equation}

\begin{definition}[The Wave Equation]
    The wave equation is a second-order linear partial differential equation that describes the propagation of waves, such as sound waves, light waves, and water waves, through a medium. In one dimension, it is given by the following PDE:
    \begin{equation}
        \frac{\partial^2 u}{\partial t^2} - c^2 \frac{\partial^2 u}{\partial x^2} = 0
    \end{equation}
    where \( u(x,t) \) is the wave function, \( c \) is the speed of the wave in the medium, \( x \) is the spatial coordinate, and \( t \) is time.
\end{definition}

\subsection{Simple Harmonic Motion}
\begin{definition}[Simple Harmonic Motion]
    You should have leaned the Hooke's law and Newton's second law, which gives us the equation of motion for a simple harmonic oscillator. The same with the equation (\ref{eq:hooke}), which can be rewritten as:
    \begin{equation}
        F = m\ddot{x} = -kx
    \end{equation}
    By setting $\omega^2 = \frac{k}{m}$, ageneral solution can be written as:
    \begin{equation}
        x(t) = x_0 + A_1 \cos(\omega t) + A_2 \sin(\omega t)
    \end{equation}
    where \( A \) are the constants determined by the IVP, \( \omega \) is the angular frequency, and \( \phi \) is the phase constant. $x_0$ is the equilibrium position where we generally set it to be 0. The unknown constant can be determined by knowing $x, \dot{x}$ at specific times.
\end{definition}

\begin{definition}[Period, Frequency, and Angular Frequency]
    The period \( T \) is the time it takes for one complete cycle of the motion, given by:
    \begin{equation}
        T = 2\pi \sqrt{\frac{m}{k}}
    \end{equation}
    The frequency \( f \) is the number of cycles per unit time, given by:
    \begin{equation}
        f = \frac{1}{T} = \frac{1}{2\pi} \sqrt{\frac{k}{m}}
    \end{equation}
    The angular frequency \( \omega \) is related to the frequency by:
    \begin{equation}
        \omega = 2\pi f = \sqrt{\frac{k}{m}}
    \end{equation}
    
\end{definition}

\begin{example}
    A simple harmonic oscillator consisting of mass \(m = 11.0\ \mathrm{kg}\) attached to a spring with spring constant \(k = 201\ \mathrm{N\,m^{-1}}\). At time \(t=0\ \mathrm{s}\) the oscillator is at position \(x(0) = -0.207\ \mathrm{m}\) and has velocity \(v(0) = -1.33\ \mathrm{m\,s^{-1}}\). Determine all coefficients of the equation describing the position \(x(t)\) of the oscillator as a function of time, assuming the offset is zero. 

    To solve for $A_1$ and $A_2$, while we assume $x_0 = 0$, we can use the initial conditions:
    \begin{align*}
        x(0) &= A_1 \cos(0) + A_2 \sin(0) = A_1 = -0.207\ \mathrm{m} \\
        v(0) &= -A_1 \omega \sin(0) + A_2 \omega \cos(0) = A_2 \omega = -1.33\ \mathrm{m\,s^{-1}}
    \end{align*}

    We can find $\omega$ from the given $m$ and $k$:
    $$
    \omega = \sqrt{\frac{k}{m}} = \sqrt{\frac{201\ \mathrm{N\,m^{-1}}}{11.0\ \mathrm{kg}}} \approx 4.28\ \mathrm{rad\,s^{-1}}
    $$
    Therefore, we can solve for $A_2$:
    $$
    A_2 = \frac{v(0)}{\omega} = \frac{-1.33\ \mathrm{m\,s^{-1}}}{4.28\ \mathrm{rad\,s^{-1}}} \approx -0.311\ \mathrm{m}
    $$
    Thus, the equation describing the position \(x(t)\) of the oscillator as a function of time is:
    $$
    x(t) = -0.207 \cos(4.28 t) - 0.311 \sin(4.28 t)
    $$
\end{example}
\begin{theorem}[A Trigonometric Identity]
We can also express the solution in a more compact form using a single cosine function with a phase shift:
\begin{equation}
    x(t) = A \cos(\omega t + \phi)
\end{equation}
where
\begin{subequations}
    \begin{align}
        A &= \sqrt{A_1^2 + A_2^2}, \\
        \phi &= \arctan\!\left(\frac{-A_2}{A_1}\right) 
              = \arctan\!\left(\frac{-v(0)/\omega}{x(0)}\right).
    \end{align}
\end{subequations}
\end{theorem}

\begin{proof}
Let $A = \sqrt{A_1^2 + A_2^2}$ and choose $\phi$ such that 
\[
\cos(\phi) = \frac{A_1}{A}, \qquad \sin(\phi) = -\frac{A_2}{A}.
\]
Then, we can rewrite our original solution as
\begin{align*}
    x(t) &= A_1 \cos(\omega t) - A_2 \sin(\omega t) \\
         &= A \cos(\phi) \cos(\omega t) - A \sin(\phi) \sin(\omega t) \\
         &= A \big[\cos(\phi)\cos(\omega t) - \sin(\phi)\sin(\omega t)\big] \\ 
         &= A \cos(\omega t + \phi),
\end{align*}
by the cosine addition formula.
\end{proof}

\begin{example}
    To determine the amplitude \(A\) and phase constant \(\phi\) for the oscillator in the previous example, we can use the values of \(A_1\) and \(A_2\) we found:
    \begin{align*}
        A &= \sqrt{(-0.207)^2 + (-0.311)^2} \approx 0.374\ \mathrm{m} \\
        \phi &= \arctan\!\left(\frac{-(-0.311)}{-0.207}\right) \approx 4.12\ \mathrm{rad} \quad (\text{since } A_1 < 0 \text{ and } A_2 < 0)
    \end{align*}
    Therefore, the equation describing the position \(x(t)\) of the oscillator as a function of time can also be written as:
    $$
    x(t) = 0.374 \cos(4.28 t + 4.12)
    $$
\end{example}

\begin{definition}[The Energy of a Simple Harmonic Oscillator]
    The total mechanical energy \(E\) of a simple harmonic oscillator is the sum of its kinetic energy \(K\) and potential energy \(U\). 
    \begin{equation}
        E = K + U
    \end{equation}
    First we consider the change of potential energy from a position $x_i$ to $x_f$, assuming the path is along the spring or the curve $C$ of the oscillator. The force exerted by the spring is given by Hooke's law, \( F = -kx \). The change in potential energy can be simply parametized and calculated as follows:
    \begin{equation}
        \Delta U = \int_C F\cdot ds = -\int_{x_i}^{x_f} F\,dx = \int_{x_i}^{x_f} kx\,dx = \left[\frac{1}{2}kx^2\right]_{x_i}^{x_f} = \frac{1}{2}k(x_f^2 - x_i^2)
    \end{equation}

    Therefore, the potential energy \(U\) at a position \(x\) (taking the reference point at \(x=0\)) is given by:
    \begin{equation}
        U(x) = \frac{1}{2}kx^2
    \end{equation}

    The kinetic energy \(K\) of the oscillator is given by:
    \begin{equation}
        K = \frac{1}{2}m\dot{x}^2
    \end{equation}
    Therefore, the total mechanical energy \(E\) of the simple harmonic oscillator is:
    \begin{equation}
        E = K + U = \frac{1}{2}m\dot{x}^2 + \frac{1}{2}kx^2
    \end{equation}
    The total mechanical energy \(E\) remains constant over time, as energy is conserved in the absence of non-conservative forces (like friction or air resistance).

\end{definition}

\subsection{Damped Harmonic Motion}
\begin{definition}[Damped Harmonic Motion]
    For small velocities, the drag force is approximately proportional to the velocity and acts in the opposite direction. This drag force can be modeled as \( F_d = -\gamma \dot{x} \), where \( \gamma \) is the damping coefficient. Including this drag force in the equation of motion for a harmonic oscillator leads to the damped harmonic oscillator equation (\ref{eq:damped_ho}). Which could be rewritten as:
    \begin{equation}
        \ddot{x} + \gamma \dot{x} + \omega_0^2 x = 0
    \end{equation}
    where \( \omega_0 = \sqrt{\frac{k}{m}} \) is the natural angular frequency of the undamped oscillator, and \( \gamma = \frac{b}{m} \) is the damping coefficient per unit mass.
    
    To skip the math, lets assume a solution of the form \( x(t) = e^{i\omega t} \), substituting into the differential equation gives us a formulation for \( \omega \):
    \begin{align}
        -\omega^2 - i\gamma\omega + \omega_0^2 &= 0 \nonumber \\
        \omega &= -i\frac{\gamma}{2} \pm \sqrt{\omega_0^2 - \frac{\gamma^2}{4}} \label{eq:omega_damped}
    \end{align}
    Also, we can characterize the real and imaginary parts of \( \omega \) as:
    \begin{subequations}
        \begin{equation}
            \omega_r = \mathrm{Re}(\omega) = \pm \sqrt{\omega_0^2 - \frac{\gamma^2}{4}}
        \end{equation}
        \begin{equation}
            \omega_i = \mathrm{Im}(\omega) = -\frac{\gamma}{2}
        \end{equation}
    \end{subequations}
    The general solution for the damped harmonic oscillator can be written as:
    \begin{equation}
        x(t) = \exp(\omega_i t) \exp(-i \omega_r t) = \exp\left(-\frac{\gamma}{2}t\right) \exp(\mp i \sqrt{\omega_0^2 - \frac{\gamma^2}{4}} t)
    \end{equation}
    \begin{subequations}
    \begin{itemize}
        \item \textbf{No Damping} (\( \gamma = 0 \)): The system behaves like a simple harmonic oscillator with angular frequency \( \omega_0 \). Given by:
        \begin{equation}
            z = \exp(-i\omega_0 t)
        \end{equation}
        \item \textbf{Underdamping} (\( 0 < \gamma < 2\omega_0 \)): The system oscillates with a gradually decreasing amplitude. The angular frequency of oscillation is given by \( \omega_r = \sqrt{\omega_0^2 - \frac{\gamma^2}{4}} \). Given by:
        \begin{equation}
            z = \exp\left(-\frac{\gamma}{2}t\right) \exp(-i \omega_r t)
        \end{equation}
        The trigonometric form of the solution is:
        \begin{equation}
            x(t) = A_0 \exp\left(-\frac{\gamma}{2}t\right) \cos(\omega_r t + \phi)
        \end{equation}
        where \( A_0 \) and \( \phi \) are constants determined by the initial conditions
        From this, we can derive the following cases:
        \item \textbf{Critical Damping} (\( \gamma = 2\omega_0 \)): The system returns to equilibrium as quickly as possible without oscillating. The solution is given by:
        \begin{equation}
            z = \exp(-\omega_0 t)
        \end{equation}\
        \item \textbf{Overdamping} (\( \gamma > 2\omega_0 \)): The system returns to equilibrium without oscillating, but more slowly than in the critically damped case. The solution is given by:
        \begin{equation}
            z = \exp\left(-\frac{\gamma}{2}t\right) \exp\left(\sqrt{\frac{\gamma^2}{4} - \omega_0^2} t\right)
        \end{equation}
        So the general solution is (the solution is via a substitution of $x(t) = e^{-\gamma t/2} f(t)$ into the differential equation, which resolves the ODE to a simple form):
        \begin{equation}
            x(t) = A_1 \exp\left[\left(-\frac{\gamma}{2} + \sqrt{\frac{\gamma^2}{4} - \omega_0^2}\right) t\right] + A_2 \exp\left[\left(-\frac{\gamma}{2} - \sqrt{\frac{\gamma^2}{4} - \omega_0^2}\right) t\right]
        \end{equation}
        where \( A_1 \) and \( A_2 \) are constants determined by the initial conditions.
    \end{itemize}
    \end{subequations}
\end{definition}
\chapter{Modern Physics}
\end{document}

