\documentclass[11pt]{report}
\usepackage{mathtools}
\usepackage{amsmath}
\usepackage{amssymb}
\usepackage{amsfonts}
\usepackage{amsthm}
\usepackage{xcolor}
\usepackage{graphicx}
\usepackage[top=2.0cm,bottom=2.0cm,left=2.5cm,right=2.5cm]{geometry}
\usepackage{tikz}
\usepackage{float}
\usepackage{multicol}
\usepackage{pgfplots}
\usepackage{lastpage}
\usepackage{siunitx}
\usepackage{xspace}
\usepackage[labelfont=bf]{caption}
\usepackage[hidelinks, urlcolor=blue, linkcolor=blue, colorlinks=true]{hyperref}
\usepackage[capitalize,noabbrev]{cleveref}
\usepackage[absolute]{textpos}
\usepackage{systeme}

\newcommand{\R}{\mathbb{R}}
\newcommand{\C}{\mathbb{C}}
%% define course title
\newcommand{\course}{MAT185}
\newcommand{\assignmenttitle}{Assignment 1}

%% header and footer
\firstpageheader{}{}{\textbf{{\color{red} Due:} 10:00pm, Tuesday Jan. 21, 2025}}
\firstpageheadrule
\runningheader{}{Page~\thepage~of~\numpages}{\course~--~\assignmenttitle}
\footer{}{}{}

\setlength\parindent{0pt} % no indentation in document

%% formats exam class
\qformat{\textbf{Question \thequestiontitle:}\hfill} % title of question 
\boxedpoints
\pointpoints{mark}{marks}
\pointsinrightmargin
\hpword{Marks:}
\hsword{Your score:}
\unframedsolutions
\totalformat{\boxed{\textnormal{\totalpoints~\if\totalpoints1 mark\else marks\fi}}}
\definecolor{SolutionColor}{rgb}{0,0,1}
\renewcommand{\solutiontitle}{}
\AtBeginEnvironment{solution}{\color{blue}}

% %correct choices in solution
\CorrectChoiceEmphasis{\rm}
\checkedchar{\tikz\draw[blue,fill=blue] (0,0) circle (1ex);}

% % increase distance between checkbox items
\renewcommand{\checkboxeshook}{\setlength{\itemsep}{6pt}}

%% distance between questions and parts
\renewcommand{\questionshook}{\setlength{\parsep}{10pt}}
\renewcommand{\partshook}{\setlength{\parsep}{15pt}}

%% define arrows in text
\newcommand{\arrow}{$\rightarrow$\xspace}
\newcommand{\Arrow}{$\Rightarrow$\xspace}

% % math notation:
%\veccol{1}{2}{3}
\newcommand{\veccol}[3]{
    \begin{bmatrix}
        #1\\
        #2\\
        #3\\
    \end{bmatrix}}
  
%\vecrow{1}{2}{3}
\newcommand{\vecrow}[3]{\left[#1~#2~ #3\right]}

%\matrixTwo{1}{2}{3}{4}
\newcommand{\matrixTwo}[4]{\left[\begin{array}{cc}#1&#2\\#3&#4\end{array}\right]}

% \matrixThree{1}{2}{3}{4}{5}{6}{7}{8}{9}
\newcommand{\matrixThree}[9]{\left[\begin{array}{ccc}#1&#2&#3\\#4&#5&#6\\#7&#8&#9\end{array}\right]}

%\matrixCorner{1}{2}{3}{4}
\newcommand{\matrixCorner}[4]{\left[\begin{array}{ccc}#1& \cdots&#2\\ \vdots & \ddots & \vdots\\#3&
      \cdots&#4\end{array}\right]}

% \nR
\newcommand{\nR}{{}^{n}\mathbb{R}}
% \Rn
\newcommand{\Rn}{\mathbb{R}^{n}}
% \nRn
\newcommand{\nRn}{{}^{n}\mathbb{R}^{n}}
% \nRm
\newcommand{\nRm}{{}^{n}\mathbb{R}^{m}}
% \nRm
\newcommand{\mRn}{{}^{m}\mathbb{R}^{n}}
% \mRm
\newcommand{\mRm}{{}^{m}\mathbb{R}^{m}}        

% \u
\renewcommand{\u}{{\bf u}}      
% \v
\renewcommand{\v}{{\bf v}}      
% \w
\newcommand{\w}{{\bf w}}    
% \V
\newcommand{\V}{{\bf V}}                   
       
%% define abbreviations
\newcommand{\row}{\operatorname{row}\,}
\newcommand{\col}{\operatorname{col}\,}
\renewcommand{\dim}{\operatorname{dim}\,}
\renewcommand{\span}{\operatorname{span}\,}
\newcommand{\rank}{\operatorname{rank}\,}
\renewcommand{\ker}{\operatorname{ker}\,}
\newcommand{\nul}{\operatorname{null}\,}
\renewcommand{\det}{\operatorname{det}\,}
\newcommand{\adj}{\operatorname{adj}\,}

\usepackage{xcolor}
% Sean's original colours:
%\definecolor{dkrgreen}{rgb}{0.1, 0.4, 0.3} 
\definecolor{dkrgreen}{HTML}{009988} % this is the color-blind friendly teal from below
%\definecolor{dkred}{rgb}{0.8, 0.05, 0.05} 
\definecolor{dkred}{HTML}{EE3377}  % this is the colour-blind friendly magenta from below
%\definecolor{orange}{rgb}{0.8, 0.33, 0.0}
%\definecolor{goldenrod}{rgb}{0.85, 0.65, 0.13}
\definecolor{blue}{HTML}{1965B0} % this is the colour-blind friendly blue from below
%
% colour-blind-friendly colours from https://personal.sron.nl/~pault/
\definecolor{tolBlue}{HTML}{1965B0}
\definecolor{tolMedBlue}{HTML}{5289C7}
\definecolor{tolLightBlue}{HTML}{7BAFDE} 
\definecolor{tolRed}{HTML}{E8601C} 
\definecolor{tolYellow}{HTML}{F6C141}
\definecolor{tolTeal}{HTML}{009988}
%\definecolor{tolBlue}{HTML}{0077BB} 
\definecolor{tolCyan}{HTML}{33BBEE}
\definecolor{tolTeal}{HTML}{009988} 
\definecolor{tolOrange}{HTML}{EE7733} 
%\definecolor{tolRed}{HTML}{CC3311} 
\definecolor{tolMagenta}{HTML}{EE3377} 
\definecolor{tolGrey}{HTML}{BBBBBB}

%%% This command makes a framed box of a chosen height.
\newcommand{\makenonemptybox}[2]{%
\par\nobreak\vspace{\ht\strutbox}\noindent
\setlength{\fboxrule}{0pt} % set this to 0pt to make invisible
\fbox{%
\parbox[c][#1][t]{\dimexpr\linewidth-2\fboxsep}{
  \hrule width \hsize height 0pt
  \vspace{-0.6cm}
  \color{SolutionColor}#2\color{black}
 }%
}%
}


\begin{document}
\thispagestyle{empty}
{\LARGE \bf AER 210 Lecture Notes}\\
{\large Hei Shing Cheung}\\
Vector Calculus \& Fluid Mechanics, Fall 2025 \hfill AER210\\
\\
The up-to-date version of this document can be found at \url{https://github.com/HaysonC/skulenotes}\\

\chapter{Vector Calculus}

% set section counter from 15 
\setcounter{section}{14}
\paragraph{Note} the section numbering is based on Stewart's book.
\section{Double and Triple Integrals}
\begin{definition}[Double Integral]
    Let $f(x,y)$ be a function defined on a closed and bounded region $R$ in the $xy$-plane. The double integral of $f$ over $R$ is denoted by
    \begin{equation}
        \iint_R f(x,y) \, dA = \iint_R f(x,y) \, dA
    \end{equation}
    where $dA$ represents an infinitesimal area element in the region $R$. The double integral can be interpreted as the volume under the surface defined by $z = f(x,y)$ over the region $R$.
\end{definition}

\subsection{Double Integrals in a Rectangular Region}

 By the point of seeing this note, you should be familiar with the simple case of rectangular, simple cases are provided as examples:

\begin{example}
    Find the area under the quadric surface $z = 16 - x^2 - y^2$ over the square region $R = \{ (x,y) \mid 0 \le x \le 2, 0 \le y \le 2\}$.

    \textbf{Note} We would have to ensure that the surface is above the $xy$-plane in the region of interest, which is true in this case.

    We can set up the double integral as follows:
    $$
        \iint_R (16 - x^2 - y^2) \, dA = \int_0^2 \int_0^2 (16 - x^2
        - y^2) \, dy \, dx
    $$
    First, we integrate with respect to $y$:
    $$
        \int_0^2 (16 - x^2 - y^2) \, dy = \left[ 16y - x^2y - \frac{y^3}{3} \right]_0^2 = 32 - 2x^2 - \frac{8}{3} = \frac{88}{3} - 2x^2
    $$
    Next, we integrate with respect to $x$:
    $$
        \int_0^2 \left( \frac{88}{3} - 2x^2 \right) \, dx = \left[ \frac{88}{3}x - \frac{2x^3}{3} \right]_0^2 = \frac{176}{3} - \frac{16}{3} = \frac{160}{3}
    $$
    Therefore, the area under the surface is $\frac{160}{3}$.

\end{example}

\begin{example}
    Evaluate the double integral of $f(x,y) = x - 3y^2$ over the rectangular region $R = \{ (x,y) \mid 0 \le x \le 2, 1 \le y \le 2\}$.

    We can set up the double integral as follows:
    $$
        \iint_R (x - 3y^2) \, dA = \int_0^2 \int_1^2 (x - 3y^2) \, dy \, dx
    $$
    First, we integrate with respect to $y$:
    $$
        \int_1^2 (x - 3y^2) \, dy = \left[ xy - y^3 \right]_1^2 = 2x - 8 - (x - 1) = x - 7
    $$
    Next, we integrate with respect to $x$:
    $$
        \int_0^2 (x - 7) \, dx = \left[ \frac{x^2}{2} - 7x \right]_0^2 = \left( 2 - 14 \right) - 0 = -12
    $$
    Therefore, the value of the double integral is $-12$.
\end{example}

\begin{theorem}
    If the integrand function $f(x,y)$ is seperable, i.e., $f(x,y) = g(x)h(y)$, then the double integral can be computed as follows:
    \begin{equation}
        \iint_R f(x,y) \, dA = \left( \int_a^b g(x) \, dx \right) \left( \int_c^d h(y) \, dy \right)
    \end{equation}
    where $R = \{ (x,y) \mid a \le x \le b, c \le y \le d \}$.
\end{theorem}
\begin{proof}
    \textbf{Sketch:} $h(y)$ is a constant when integrating with respect to $x$, and vice versa.
\end{proof}

\begin{example}
    Let $f(x,y) = \sin{x} \cos{y}$ and $R = \{ (x,y) \mid 0 \le x \le \frac{\pi}{2}, 0 \le y \le \frac{\pi}{2} \}$. Evaluate the double integral $\iint_R f(x,y) \, dA$.

    Since $f(x,y)$ is seperable, we can write:
    $$
        \iint_R f(x,y) \, dA = \left( \int_0^{\frac{\pi}{2}} \sin{x} \, dx \right) \left( \int_0^{\frac{\pi}{2}} \cos{y} \, dy \right)
    $$
    Evaluating each integral separately gives 1 for both, so the final result is: $1 \times 1 = 1$.
\end{example}
\subsection{Double Integrals in General Regions}
\paragraph{Types of Regions} When the region $R$ is not rectangular, we can still compute the double integral by expressing the region in terms of inequalities. There are three common types of regions:
\begin{definition}[Type I Region]
    A region $R$ is called a Type I region if it can be described by the inequalities:
    $$
        a \le x \le b, \quad g_1(x) \le y \le g_2(x)
    $$
    where $g_1(x)$ and $g_2(x)$ are continuous functions on the interval $[a,b]$. Then, to evaluate the double integral over a Type I region for a continuous function $f(x,y)$, we set up the integral as follows:

    \begin{equation}
        \iint_R f(x,y) \, dA = \int_a^b \int_{g_1(x)}^{g_2(x)} f(x,y) \, dy \, dx
    \end{equation}

    \textbf{Integral Order:} Integrate with respect to $y$ first, then $x$.

    \textbf{Intuition:} As we traverse the outer part ($x$), we are summing up vertical slices (in $y$), and the bounds of those slices depend on $x$ and changes.
\end{definition}

\begin{definition}[Type II Region]
    Type II reigion is similar to Type I, but the roles of $x$ and $y$ are swapped. A region $R$ is called a Type II region if it can be described by the inequalities:
    $$
        c \le y \le d, \quad h_1(y) \le x \le h_2(y)
    $$
    where $h_1(y)$ and $h_2(y)$ are continuous functions on the interval $[c,d]$. Then, to evaluate the double integral over a Type II region for a continuous function $f(x,y)$, we set up the integral as follows: 
    \begin{equation}
        \iint_R f(x,y) \, dA = \int_c^d \int_{h_1(y)}^{h_2(y)} f(x,y) \, dx \, dy
    \end{equation}

    The integral order and intuition is mirrored from Type I, but we are summing up horizontal slices (in $x$), and the bounds of those slices depend on $y$ and changes.
\end{definition}

\begin{definition}[Type III Region]
    A region $R$ is called a Type III region if it can be described as the union of a finite number of Type I and Type II regions. To evaluate the double integral over a Type III region for a continuous function $f(x,y)$, we can break down the integral into separate integrals over each Type I or Type II subregion and sum them up:
    \begin{equation}
        \iint_R f(x,y) \, dA = \sum_{i=1}^n \iint_{R_i} f(x,y) \, dA
    \end{equation}
    where each $R_i$ is either a Type I or Type II region. And that:
    $$
        \bigcup_{i=1}^n R_i = R \quad \text{and} \quad R_i \cap R_j = \emptyset \text{ for } i \neq j
    $$
    This approach allows us to handle more complex regions by breaking them down into simpler parts.
\end{definition}

\begin{example}
    Find the volume of the solid that lies under the paraboloid $z = f(x,y) = x^2 + y^2$ and above the region $R$ bounded
     by $y = 2x$ and $y = x^2$.

     First, you would sketch the region to understand its shape and boundaries at Figure \ref{fig:region}.

    \begin{figure}[h]
        \centering
        \begin{tikzpicture}
            \begin{axis}[
                axis lines = middle,
                xlabel = $x$,
                ylabel = $y$,
                xmin = -1, xmax = 3,
                ymin = -1, ymax = 5,
                xtick = {0,1,2,3},
                ytick = {0,1,2,3,4},
                grid = both,
                minor tick num = 1,
                domain = -1:3,
            ]
            \addplot[blue, thick, name path=twox, name path global=twox] {2*x};
            \addplot[red, thick, name path=xsq, name path global=xsq] {x^2};
            \addplot[fill=gray, opacity=0.5] fill between[of=twox and xsq, soft clip={domain=0:2}];          \end{axis}
        \end{tikzpicture}
        \caption{Region bounded by $y = 2x$ and $y = x^2$} \label{fig:region}
    \end{figure}

    We can tell that this is a Type I regionwhere $0 \le x \le 2$, and $x^2 \le y \le 2x$. Thus, we can set up the double integral as follows:
    $$
        \iint_R (x^2 + y^2) \, dA = \int_0^2 \int_{x^2}^{2x} (x^2 + y^2) \, dy \, dx
    $$
    First, we integrate with respect to $y$:
    $$
        \int_{x^2}^{2x} (x^2 + y^2) \, dy = \left[ x^2y + \frac{y^3}{3} \right]_{y=x^2}^{y=2x} = 2x^3 + \frac{8x^3}{3} - x^4 - \frac{x^6}{3} = \frac{14x^3}{3} - x^4 - \frac{x^6}{3}
    $$
    Next, we integrate with respect to $x$:
    $$
        \int_0^2 \left( \frac{14x^3}{3} - x^4 - \frac{x^6}{3} \right) \, dx = \left[ \frac{14x^4}{12} - \frac{x^5}{5} - \frac{x^7}{21} \right]_0^2 = \frac{216}{35}
    $$
    Therefore, the volume of the solid is $\frac{216}{35}$.

\end{example}

\begin{example}
    Consider the above example, but we want to set it up as a Type II region. The region $R$ can be described by $0 \le y \le 4$, and $\frac{y}{2} \le x \le \sqrt{y}$. Thus, we can set up the double integral as follows:
    $$
        \iint_R (x^2 + y^2) \, dA = \int_0^4 \int_{\frac{y}{2}}^{\sqrt{y}} (x^2 + y^2) \, dx \, dy
    $$
    First, we integrate with respect to $x$:
    $$
        \int_{\frac{y}{2}}^{\sqrt{y}} (x^2 + y^2) \, dx = \left[ \frac{x^3}{3} + y^2x \right]_{x=\frac{y}{2}}^{x=\sqrt{y}} = \frac{y^{\frac{3}{2}}}{3
        } + y^{\frac{5}{2}} - \frac{y^3}{24} - \frac{y^3}{2} = \frac{y^{\frac{3}{2}}}{3} + y^{\frac{5}{2}} - \frac{13y^3}{24}
    $$
    Next, we integrate with respect to $y$:
    $$
        \int_0^4 \left( \frac{y^{\frac{3}{2}}}{3} + y^{\frac{5}{2}} - \frac{13y^3}{24} \right) \, dy = \left[ \frac{2y^{\frac{5}{2}}}{15} + \frac{2y^{\frac{7}{2}}}{7} - \frac{13y^4}{96} \right]_0^4 = \frac{216}{35}
    $$
    Therefore, the volume of the solid is $\frac{216}{35}$, which is consistent with the previous result. This is also consistent with Fubini's Theorem.
\end{example}

\begin{example}
    Integrate the surface given by $z = e^{x^2}$ over the triangular region with vertices at $(0,0)$, $(1,0)$, and $(1,1)$. We can describe the region as either a Type I or Type II region:

    (\xmark) Here, we will describe it as a Type II regionwhere $0 \le y \le 1$, and $y \le x \le 1$. Thus, we can set up the double integral as follows:
    $$
        \iint_R e^{x^2} \, dA = \int_0^1 \int_y^1 e^{x^2} \, dx \, dy
    $$
    We can tell that $e^{x^2}$ does not have an elementary antiderivative, so we cannot integrate with respect to $x$ directly. 
    
    (\cmark) However, we can change the order of integration to make it a Type I regionwhere $0 \le x \le 1$, and $0 \le y \le x$. Thus, we can set up the double integral as follows:
    $$
        \iint_R e^{x^2} \, dA = \int_0^1 \int_0^x e^{x^2} \, dy \, dx
    $$
    First, we integrate with respect to $y$:
    $$
        \int_0^x e^{x^2} \, dy = \left[ y e^{x^2} \right]_0^x = xe^{x^2}
    $$
    Next, we integrate with respect to $x$:
    $$
        \int_0^1 xe^{x^2} \, dx
    $$
    This is now obvious, a simple $u$-substitution with $u = x^2$, $du = 2x \, dx$:
    $$
        \int_0^1 xe^{x^2} \, dx = \frac{1}{2} \int_0^1 e^u \, du = \frac{1}{2} \left[ e^u \right]_0^1 = \frac{e - 1}{2}
    $$
    Therefore, the value of the double integral is $\frac{e - 1}{2}$. 

    \textbf{Intuition} When the integrand is difficult to integrate with respect to one variable, consider changing the order of integration. You should be able to tell that $e^{x^2}$ has no elementary antiderivative, so you would have ruled out integrating with respect to $x$ first.
\end{example}

\subsection{Formal Definition of Double Integrals}
There is two definitions of double integrals in this course, due to the discrepancy between Stewart's book and the lectures. 

\begin{shaded}
    \subsubsection*{\textbf{Review.  } Formal Definition of Definite Integral (Single Variable)}

    Consider $y = f(x) \ge 0$ on the interval $x \in [a,b]$. We divide the interval into $n$ subintervals of equal width $\Delta x = \frac{b-a}{n}$, and let $x_i^*$ be a sample point in the $i$-th subinterval. The Riemann sum is given by:
    $$
        S_n = \sum_{i=1}^n f(x_i^*) \Delta x
    $$
    Now, for any $x_i^*$, we consider the minimum and maximum values of $f(x_i^*)$ in the $i$-th subinterval, denoted as $m_i$ and $M_i$ respectively. We can then define the lower sum $L_n$ and upper sum $U_n$ as follows:
    $$
        L_n = \sum_{i=1}^n m_i \Delta x \quad \text{and} \quad U_n = \sum_{i=1}^n M_i \Delta x
    $$
    To satisfy the squeeze theorem, for all $i$, we would need:
    $$
        \lim_{n \to \infty} M_i - m_i = \lim_{\delta x \to 0} M_i - m_i = 0
    $$
    If $f(x)$ is continuous on $[a,b]$. Then, we have:
    $$
        \lim_{n \to \infty} L_n = \lim_{n \to \infty} U_n = \int_a^b f(x) \, dx
    $$

    For the case of discontinuous functions, if the set of discontinuities has measure zero, then the function is still integrable.
\end{shaded}

\begin{definition}[Definition of Double Integral]
    Let $R$ be a rectangular region in the $xy$-plane given by $R = [a,b] \times [c,d]$. The double integral of a function $f(x,y)$ over the region $R$ is defined as:
    \begin{subequations}
    \begin{equation}
        \iint_R f(x,y) \, dA = \lim_{n \to \infty} \sum_{i=1}^n f(x_i^*,y_i^*) \Delta A_i \quad (\text{Riemann Definition})
    \end{equation}
    where $\Delta A_i$ is the area of the $i$-th subrectangle, and $(x_i^*,y_i^*)$ is a sample point in it. The limit is taken as the maximum diameter of the subrectangles approaches zero.
    \begin{equation}
        \iint_R f(x,y) \, dA = \lim_{n,m \to \infty} \sum_{i=1}^n \sum_{j=1}^m f(x_i^*,y_j^*) \Delta A_{ij} \quad (\text{Grid Formulation})
    \end{equation}
    where $\Delta A_{ij}$ is the area of the $ij$-th subrectangle, and $(x_i^*,y_j^*)$ is a sample point in it. Note that the $\Delta A_{ij}$ may be non-uniform. The limit is taken as the maximum diameter of the subrectangles approaches zero.
    \end{subequations}

    Similarly, the lower and upper sums for double integrals are:
    \begin{subequations}
    \begin{equation}
        L_n = \sum_{i=1}^n m_i \Delta A_i \quad \text{and} \quad U_n = \sum_{i=1}^n M_i \Delta A_i \quad (\text{Riemann Definition})
    \end{equation}
    \begin{equation}
        L_{n,m} = \sum_{i=1}^n \sum_{j=1}^m m_{ij} \Delta A_{ij} \quad \text{and} \quad U_{n,m} = \sum_{i=1}^n \sum_{j=1}^m M_{ij} \Delta A_{ij} \quad (\text{Grid Formulation})
    \end{equation}
    \end{subequations}
    Here, $m_{ij}$ and $M_{ij}$ are the minimum and maximum values of $f(x,y)$ in the $ij$-th subrectangle. Define $||P|| = \max ||(\Delta x_i, \Delta y_j)||$ as the maximum diameter of the subrectangles. For the squeeze theorem, we require:
    $$
        \lim_{n,m \to \infty} (M_{ij} - m_{ij}) = \lim_{||P|| \to 0} (M_{ij} - m_{ij}) = 0
    $$
    If $f(x,y)$ is continuous on $R$, then:
    $$
        \lim_{n,m \to \infty} L_{n,m} = \lim_{n,m \to \infty} U_{n,m} = \iint_R f(x,y) \, dA
    $$

    The Riemann definition and grid formulation are similar.
\end{definition}



\chapter{Fluid Mechanics}
\end{document}