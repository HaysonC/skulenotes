\documentclass[11pt]{article}
\usepackage{mathtools}
\usepackage{amsmath}
\usepackage{amssymb}
\usepackage{amsfonts}
\usepackage{amsthm}
\usepackage{xcolor}
\usepackage{graphicx}
\usepackage[top=2.0cm,bottom=2.0cm,left=2.5cm,right=2.5cm]{geometry}
\usepackage{tikz}
\usepackage{float}
\usepackage{multicol}
\usepackage{pgfplots}
\usepackage{lastpage}
\usepackage{siunitx}
\usepackage{xspace}
\usepackage[labelfont=bf]{caption}
\usepackage[hidelinks, urlcolor=blue, linkcolor=blue, colorlinks=true]{hyperref}
\usepackage[capitalize,noabbrev]{cleveref}
\usepackage[absolute]{textpos}
\usepackage{systeme}

\newcommand{\R}{\mathbb{R}}
\newcommand{\C}{\mathbb{C}}
%% define course title
\newcommand{\course}{MAT185}
\newcommand{\assignmenttitle}{Assignment 1}

%% header and footer
\firstpageheader{}{}{\textbf{{\color{red} Due:} 10:00pm, Tuesday Jan. 21, 2025}}
\firstpageheadrule
\runningheader{}{Page~\thepage~of~\numpages}{\course~--~\assignmenttitle}
\footer{}{}{}

\setlength\parindent{0pt} % no indentation in document

%% formats exam class
\qformat{\textbf{Question \thequestiontitle:}\hfill} % title of question 
\boxedpoints
\pointpoints{mark}{marks}
\pointsinrightmargin
\hpword{Marks:}
\hsword{Your score:}
\unframedsolutions
\totalformat{\boxed{\textnormal{\totalpoints~\if\totalpoints1 mark\else marks\fi}}}
\definecolor{SolutionColor}{rgb}{0,0,1}
\renewcommand{\solutiontitle}{}
\AtBeginEnvironment{solution}{\color{blue}}

% %correct choices in solution
\CorrectChoiceEmphasis{\rm}
\checkedchar{\tikz\draw[blue,fill=blue] (0,0) circle (1ex);}

% % increase distance between checkbox items
\renewcommand{\checkboxeshook}{\setlength{\itemsep}{6pt}}

%% distance between questions and parts
\renewcommand{\questionshook}{\setlength{\parsep}{10pt}}
\renewcommand{\partshook}{\setlength{\parsep}{15pt}}

%% define arrows in text
\newcommand{\arrow}{$\rightarrow$\xspace}
\newcommand{\Arrow}{$\Rightarrow$\xspace}

% % math notation:
%\veccol{1}{2}{3}
\newcommand{\veccol}[3]{
    \begin{bmatrix}
        #1\\
        #2\\
        #3\\
    \end{bmatrix}}
  
%\vecrow{1}{2}{3}
\newcommand{\vecrow}[3]{\left[#1~#2~ #3\right]}

%\matrixTwo{1}{2}{3}{4}
\newcommand{\matrixTwo}[4]{\left[\begin{array}{cc}#1&#2\\#3&#4\end{array}\right]}

% \matrixThree{1}{2}{3}{4}{5}{6}{7}{8}{9}
\newcommand{\matrixThree}[9]{\left[\begin{array}{ccc}#1&#2&#3\\#4&#5&#6\\#7&#8&#9\end{array}\right]}

%\matrixCorner{1}{2}{3}{4}
\newcommand{\matrixCorner}[4]{\left[\begin{array}{ccc}#1& \cdots&#2\\ \vdots & \ddots & \vdots\\#3&
      \cdots&#4\end{array}\right]}

% \nR
\newcommand{\nR}{{}^{n}\mathbb{R}}
% \Rn
\newcommand{\Rn}{\mathbb{R}^{n}}
% \nRn
\newcommand{\nRn}{{}^{n}\mathbb{R}^{n}}
% \nRm
\newcommand{\nRm}{{}^{n}\mathbb{R}^{m}}
% \nRm
\newcommand{\mRn}{{}^{m}\mathbb{R}^{n}}
% \mRm
\newcommand{\mRm}{{}^{m}\mathbb{R}^{m}}        

% \u
\renewcommand{\u}{{\bf u}}      
% \v
\renewcommand{\v}{{\bf v}}      
% \w
\newcommand{\w}{{\bf w}}    
% \V
\newcommand{\V}{{\bf V}}                   
       
%% define abbreviations
\newcommand{\row}{\operatorname{row}\,}
\newcommand{\col}{\operatorname{col}\,}
\renewcommand{\dim}{\operatorname{dim}\,}
\renewcommand{\span}{\operatorname{span}\,}
\newcommand{\rank}{\operatorname{rank}\,}
\renewcommand{\ker}{\operatorname{ker}\,}
\newcommand{\nul}{\operatorname{null}\,}
\renewcommand{\det}{\operatorname{det}\,}
\newcommand{\adj}{\operatorname{adj}\,}

\usepackage{xcolor}
% Sean's original colours:
%\definecolor{dkrgreen}{rgb}{0.1, 0.4, 0.3} 
\definecolor{dkrgreen}{HTML}{009988} % this is the color-blind friendly teal from below
%\definecolor{dkred}{rgb}{0.8, 0.05, 0.05} 
\definecolor{dkred}{HTML}{EE3377}  % this is the colour-blind friendly magenta from below
%\definecolor{orange}{rgb}{0.8, 0.33, 0.0}
%\definecolor{goldenrod}{rgb}{0.85, 0.65, 0.13}
\definecolor{blue}{HTML}{1965B0} % this is the colour-blind friendly blue from below
%
% colour-blind-friendly colours from https://personal.sron.nl/~pault/
\definecolor{tolBlue}{HTML}{1965B0}
\definecolor{tolMedBlue}{HTML}{5289C7}
\definecolor{tolLightBlue}{HTML}{7BAFDE} 
\definecolor{tolRed}{HTML}{E8601C} 
\definecolor{tolYellow}{HTML}{F6C141}
\definecolor{tolTeal}{HTML}{009988}
%\definecolor{tolBlue}{HTML}{0077BB} 
\definecolor{tolCyan}{HTML}{33BBEE}
\definecolor{tolTeal}{HTML}{009988} 
\definecolor{tolOrange}{HTML}{EE7733} 
%\definecolor{tolRed}{HTML}{CC3311} 
\definecolor{tolMagenta}{HTML}{EE3377} 
\definecolor{tolGrey}{HTML}{BBBBBB}

%%% This command makes a framed box of a chosen height.
\newcommand{\makenonemptybox}[2]{%
\par\nobreak\vspace{\ht\strutbox}\noindent
\setlength{\fboxrule}{0pt} % set this to 0pt to make invisible
\fbox{%
\parbox[c][#1][t]{\dimexpr\linewidth-2\fboxsep}{
  \hrule width \hsize height 0pt
  \vspace{-0.6cm}
  \color{SolutionColor}#2\color{black}
 }%
}%
}


\begin{document}
\thispagestyle{empty}
{\LARGE \bf MIE 286 Lecture Notes}\\
{\large Hei Shing Cheung}\\
Probability and Statistics, Winter 2025 \hfill MIE286\\
\\
The up-to-date version of this document can be found at \url{https://github.com/HaysonC/skulenotes}\\
\section{Random Variables}
\begin{definition}[Random Variable]
A random variable is a variable whose possible values are numerical outcomes of a random phenomenon. There are two main types of random variables:
\begin{itemize}
    \item Discrete Random Variable: Takes on a countable number of distinct values. Examples include the outcome of rolling a die or the number of heads in a series of coin flips.
    \item Continuous Random Variable: Takes on an infinite number of possible values within a given range. Examples include the height of individuals or the time it takes to complete a task.
    \item Mixed Random Variable: Exhibits properties of both discrete and continuous random variables. It can take on specific discrete values as well as a continuous range of values.
\end{itemize}
\end{definition}

\begin{definition}[Probability Mass Function (PMF)]
A Probability Mass Function (PMF) is a function that gives the probability that a discrete random variable is exactly equal to some value. The PMF must satisfy the following properties:
\begin{itemize}
    \item Non-negativity: \( P(X = x) \geq 0 \) for all \( x \).
    \item Normalization: The sum of the probabilities over all possible values of the random variable must equal 1:
    \[ \sum_{x} P(X = x) = 1 \]
\end{itemize}
\end{definition}

\begin{definition}[Probability Density Function (PDF)] 
A Probability Density Function (PDF) is a function that describes the likelihood of a continuous random variable to take on a particular value. The PDF must satisfy the following properties:
\begin{itemize}
    \item Non-negativity: \( f(x) \geq 0 \) for all \( x \).
    \item Normalization: The total area under the PDF curve must equal 1:
    \[ \int_{-\infty}^{\infty} f(x) \, dx = 1 \]
\end{itemize}
\end{definition}
% Common distributions, visualizations, experiments, sample space
\section{Common Distributions and Visualization}
\subsection{Uniform Distribution}
\begin{definition}[Uniform Distribution]
Discrete uniform on a finite set $\{x_1,\dots,x_n\}$: $P(X=x_i)=1/n$ for each $i$.
Continuous uniform on an interval $[a,b]$: the density is
$$
f(x)=\frac{1}{b-a},\qquad a\le x\le b,
$$
with mean $\mu=(a+b)/2$ and variance $\sigma^2=(b-a)^2/12$.
\end{definition}

\subsection{Normal Distribution}
\begin{definition}[Normal Distribution]
A random variable $X$ is normal with mean $\mu$ and variance $\sigma^2$, written $X\sim\mathcal{N}(\mu,\sigma^2)$, if its density is
$$  
f(x)=\frac{1}{\sigma\sqrt{2\pi}}\exp\left(-\frac{(x-\mu)^2}{2\sigma^2}\right).
$$
The distribution is symmetric about $\mu$, and the standard normal is $\mathcal{N}(0,1)$.
\end{definition}

\subsection{Addign Distributions}
\paragraph{} We can add random variables or constants. For example, we can do $Y = 2 + \mathrm{Uni}(0,1)$ or $Z = \mathrm{Norm}(0,1) + \mathrm{Norm}(0,1)$. The resulting distribution can be found by convolution of the original distributions.

\subsection{Symmetric Distributions}
A distribution is symmetric about a point $c$ if for all $x$ the density or mass satisfies $f(c+x)=f(c-x)$. Symmetry implies the mean (when it exists) equals the center $c$.

\subsection{Stem-and-Leaf Plot}
A stem-and-leaf plot is a simple textual visualization which preserves raw data while showing shape. Example for the data
$$\{12,14,15,17,21,22,22,24,31\}$$
is shown with stems (tens) and leaves (ones):
\begin{verbatim}
1 | 2 4 5 7
2 | 1 2 2 4
3 | 1
\end{verbatim}
This quickly reveals distributional shape and outliers.

\subsection{Experiments and Sample Space}
\begin{definition}[Experiment]
An experiment is a process that leads to the occurrence of one and only one of several possible outcomes. Simply, it generates data. 
\end{definition}
\begin{definition}[Sample Space]
The sample space, denoted by $\Omega$, is the set of all possible outcomes of an experiment. Each outcome in the sample space is called a sample point.
\end{definition}
\section{Sets and Sample Space}
The sample space $\Omega$ is a set and events are subsets of $\Omega$. For sets $A,B\subseteq\Omega$ we define:
\begin{itemize}
    \item \textbf{Union:} $A\cup B=\{x:\;x\in A\text{ or }x\in B\}$.
    \item \textbf{Intersection:} $A\cap B=\{x:\;x\in A\text{ and }x\in B\}$.
    \item \textbf{Complement:} $A^c=\Omega\setminus A=\{x:\;x\notin A\}$.
    \item \textbf{Set difference:} $A\setminus B=\{x:\;x\in A,\;x\notin B\}$.
    \item \textbf{Disjoint:} $A$ and $B$ are disjoint if $A\cap B=\varnothing$.
    \item \textbf{Subset:} $A\subseteq B$ means every element of $A$ is also in $B$.
\end{itemize}

Basic probability properties follow:
\begin{itemize}
    \item $P(\varnothing)=0$, $P(\Omega)=1$.
    \item For any event $A$, $P(A^c)=1-P(A)$.
    \item For any events $A,B$, $P(A\cup B)=P(A)+P(B)-P(A\cap B)$.
\end{itemize}

De Morgan's laws: $(A\cup B)^c=A^c\cap B^c$ and $(A\cap B)^c=A^c\cup B^c$.

\end{document}