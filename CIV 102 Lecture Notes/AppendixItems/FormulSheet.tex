\section{Formula Sheet}
\setcounter{equation}{0}
% New command for creating boxes with black text
\definecolor{lightblue}{rgb}{153, 3, 32}
\newcommand{\mybox}[2]{
    \begin{tcolorbox}[colback=lightblue!5!white,colframe=lightblue!75!black,boxsep=1pt,arc=0pt,outer arc=0pt,title={\textcolor{black}{#1}}]
        \textcolor{black}{#2}
    \end{tcolorbox}
}

% Start the document

\begin{multicols}{2}
\mybox{Geometry}{
Area of a Circle:
\begin{equation}
    A = \pi r^2
\end{equation}
Circumference of a Circle (in terms of radians):
\begin{equation}
    C = 2\pi r \quad (\text{radians in a full circle})
\end{equation}
Pythagorean Theorem:
\begin{equation}
    c^2 = a^2 + b^2
\end{equation}
Area of a Triangle:
\begin{equation}
    A = \frac{1}{2} b h
\end{equation}
Arc Length of a Circle (radians):
\begin{equation}
    s = r \theta \quad (\text{where } \theta \text{ is in radians})
\end{equation}
Area of a Sector (radians):
\begin{equation}
    A = \frac{1}{2} r^2 \theta \quad (\text{where } \theta \text{ is in radians})
\end{equation}
Volume of a Sphere:
\begin{equation}
    V = \frac{4}{3} \pi r^3
\end{equation}
}

\mybox{Forces}{
Force:
\begin{equation}
    \Vec{F}=m\Vec{a}
\end{equation}
Friction:
\begin{equation}
    F_f\le \mu R
\end{equation}
Moment:
\begin{equation}
    \mu = \text{moment} =\Vec{F} \times \Vec{d}_\perp
\end{equation}
Load
\begin{equation}
     F=\int wdl 
\end{equation}
}

\mybox{Momentum and Impulse}{
    \begin{equation}
    \Delta P = \Delta mv = F\Delta T
    \end{equation}
}

\mybox{Energy}{
    Moment-Energy
    \begin{equation}
        E = \frac{p^2}{2m}
    \end{equation}
    Kinetic, SHM
    \begin{equation}
        E_k = \frac{1}{2}mv^2 = \frac{1}{2}m\omega^2(x_0^2-x^2) = -\frac{1}{2}kx^2
    \end{equation}
    Strain Energy
    \begin{equation}
        \int \sigma d\epsilon
    \end{equation}
}
\mybox{Materials}{
Stress:
\begin{equation}
    \sigma = \frac{F}{A}
\end{equation}
Strain (linear):
\begin{equation}
    \epsilon = \frac{\Delta L}{L_0}
\end{equation}
Young's Modulus (Elastic Modulus):
\begin{equation}
    E = \frac{\sigma}{\epsilon}
\end{equation}
}
\mybox{Materials}{
Spring Constant
\begin{equation}
    F = -kx
\end{equation}
\begin{equation}
    k = \frac{EA}{\L}
\end{equation}
}
\end{multicols}