\usepackage{mathtools}
\usepackage{amsmath}
\usepackage{amssymb}
\usepackage{amsfonts}
\usepackage{amsthm}
\usepackage{xcolor}
\usepackage{graphicx}
\usepackage[top=2.0cm,bottom=2.0cm,left=2.5cm,right=2.5cm]{geometry}
\usepackage{tikz}
\usepackage{float}
\usepackage{multicol}
\usepackage{pgfplots}
\usepackage{lastpage}
\usepackage{siunitx}
\usepackage{xspace}
\usepackage[labelfont=bf]{caption}
\usepackage[hidelinks, urlcolor=blue, linkcolor=blue, colorlinks=true]{hyperref}
\usepackage[capitalize,noabbrev]{cleveref}
\usepackage[absolute]{textpos}
\usepackage{systeme}

\newcommand{\R}{\mathbb{R}}
\newcommand{\C}{\mathbb{C}}
%% define course title
\newcommand{\course}{MAT185}
\newcommand{\assignmenttitle}{Assignment 1}

%% header and footer
\firstpageheader{}{}{\textbf{{\color{red} Due:} 10:00pm, Tuesday Jan. 21, 2025}}
\firstpageheadrule
\runningheader{}{Page~\thepage~of~\numpages}{\course~--~\assignmenttitle}
\footer{}{}{}

\setlength\parindent{0pt} % no indentation in document

%% formats exam class
\qformat{\textbf{Question \thequestiontitle:}\hfill} % title of question 
\boxedpoints
\pointpoints{mark}{marks}
\pointsinrightmargin
\hpword{Marks:}
\hsword{Your score:}
\unframedsolutions
\totalformat{\boxed{\textnormal{\totalpoints~\if\totalpoints1 mark\else marks\fi}}}
\definecolor{SolutionColor}{rgb}{0,0,1}
\renewcommand{\solutiontitle}{}
\AtBeginEnvironment{solution}{\color{blue}}

% %correct choices in solution
\CorrectChoiceEmphasis{\rm}
\checkedchar{\tikz\draw[blue,fill=blue] (0,0) circle (1ex);}

% % increase distance between checkbox items
\renewcommand{\checkboxeshook}{\setlength{\itemsep}{6pt}}

%% distance between questions and parts
\renewcommand{\questionshook}{\setlength{\parsep}{10pt}}
\renewcommand{\partshook}{\setlength{\parsep}{15pt}}

%% define arrows in text
\newcommand{\arrow}{$\rightarrow$\xspace}
\newcommand{\Arrow}{$\Rightarrow$\xspace}

% % math notation:
%\veccol{1}{2}{3}
\newcommand{\veccol}[3]{
    \begin{bmatrix}
        #1\\
        #2\\
        #3\\
    \end{bmatrix}}
  
%\vecrow{1}{2}{3}
\newcommand{\vecrow}[3]{\left[#1~#2~ #3\right]}

%\matrixTwo{1}{2}{3}{4}
\newcommand{\matrixTwo}[4]{\left[\begin{array}{cc}#1&#2\\#3&#4\end{array}\right]}

% \matrixThree{1}{2}{3}{4}{5}{6}{7}{8}{9}
\newcommand{\matrixThree}[9]{\left[\begin{array}{ccc}#1&#2&#3\\#4&#5&#6\\#7&#8&#9\end{array}\right]}

%\matrixCorner{1}{2}{3}{4}
\newcommand{\matrixCorner}[4]{\left[\begin{array}{ccc}#1& \cdots&#2\\ \vdots & \ddots & \vdots\\#3&
      \cdots&#4\end{array}\right]}

% \nR
\newcommand{\nR}{{}^{n}\mathbb{R}}
% \Rn
\newcommand{\Rn}{\mathbb{R}^{n}}
% \nRn
\newcommand{\nRn}{{}^{n}\mathbb{R}^{n}}
% \nRm
\newcommand{\nRm}{{}^{n}\mathbb{R}^{m}}
% \nRm
\newcommand{\mRn}{{}^{m}\mathbb{R}^{n}}
% \mRm
\newcommand{\mRm}{{}^{m}\mathbb{R}^{m}}        

% \u
\renewcommand{\u}{{\bf u}}      
% \v
\renewcommand{\v}{{\bf v}}      
% \w
\newcommand{\w}{{\bf w}}    
% \V
\newcommand{\V}{{\bf V}}                   
       
%% define abbreviations
\newcommand{\row}{\operatorname{row}\,}
\newcommand{\col}{\operatorname{col}\,}
\renewcommand{\dim}{\operatorname{dim}\,}
\renewcommand{\span}{\operatorname{span}\,}
\newcommand{\rank}{\operatorname{rank}\,}
\renewcommand{\ker}{\operatorname{ker}\,}
\newcommand{\nul}{\operatorname{null}\,}
\renewcommand{\det}{\operatorname{det}\,}
\newcommand{\adj}{\operatorname{adj}\,}

\usepackage{xcolor}
% Sean's original colours:
%\definecolor{dkrgreen}{rgb}{0.1, 0.4, 0.3} 
\definecolor{dkrgreen}{HTML}{009988} % this is the color-blind friendly teal from below
%\definecolor{dkred}{rgb}{0.8, 0.05, 0.05} 
\definecolor{dkred}{HTML}{EE3377}  % this is the colour-blind friendly magenta from below
%\definecolor{orange}{rgb}{0.8, 0.33, 0.0}
%\definecolor{goldenrod}{rgb}{0.85, 0.65, 0.13}
\definecolor{blue}{HTML}{1965B0} % this is the colour-blind friendly blue from below
%
% colour-blind-friendly colours from https://personal.sron.nl/~pault/
\definecolor{tolBlue}{HTML}{1965B0}
\definecolor{tolMedBlue}{HTML}{5289C7}
\definecolor{tolLightBlue}{HTML}{7BAFDE} 
\definecolor{tolRed}{HTML}{E8601C} 
\definecolor{tolYellow}{HTML}{F6C141}
\definecolor{tolTeal}{HTML}{009988}
%\definecolor{tolBlue}{HTML}{0077BB} 
\definecolor{tolCyan}{HTML}{33BBEE}
\definecolor{tolTeal}{HTML}{009988} 
\definecolor{tolOrange}{HTML}{EE7733} 
%\definecolor{tolRed}{HTML}{CC3311} 
\definecolor{tolMagenta}{HTML}{EE3377} 
\definecolor{tolGrey}{HTML}{BBBBBB}

%%% This command makes a framed box of a chosen height.
\newcommand{\makenonemptybox}[2]{%
\par\nobreak\vspace{\ht\strutbox}\noindent
\setlength{\fboxrule}{0pt} % set this to 0pt to make invisible
\fbox{%
\parbox[c][#1][t]{\dimexpr\linewidth-2\fboxsep}{
  \hrule width \hsize height 0pt
  \vspace{-0.6cm}
  \color{SolutionColor}#2\color{black}
 }%
}%
}


\begin{document}
\thispagestyle{empty}
{\LARGE \bf CIV 102 Lecture Notes}\\
{\large Hei Shing Cheung}\\
Structures and Materials, Fall 2024 \hfill CIV102\\
\begin{center}
\textit{"Structural Engineering is the art and science of designing and molding structures with economy and elegance so that they can safely resist the force that they are subjected''} \\ - Prof. \textsc{Evan Bentz, 2024}
\end{center}
\vspace{10pt}
\section{Force, Moment, and Geometry} 
    \paragraph{Moment for One Force} The moment due to only one force is:
    \begin{equation}
        \mu = \mathrm{moment} = \Vec{F}\times\Vec{d_\perp}
    \end{equation}
Where:
\begin{equation*}
\begin{split}
\Vec{d_\perp} = \text{The perpendicular displacement to the center of rotation}
\end{split}
\end{equation*}
\paragraph{Centroid} The centroid of a shape with multiple geometries is calculated by: 
\begin{equation}
\bar{y} = \frac{\sum_i A_i  y_i}{\sum_i A_i}
\end{equation}
\paragraph{Parallel axis Theorem} The moment of inertia is calculated by the following
\begin{equation}
    I = \sum_i I_{i} + \sum_i A_i d_i^2
\end{equation}
Where:
\begin{equation*}
\begin{split}
d_i = \text{The displacement of $\bar{y_i}$ to $\bar{y}$}
\end{split}
\end{equation*}
\paragraph{First Moment of Area (Q)} It is expressed as.
\begin{equation}
    Q = \sum_i A_i \cdot d_i
\end{equation}
\section{Truss}
\paragraph{Area} Design area against tension/compression by:
\begin{equation}
    A \geq \frac{2F}{\sigma_y}
\end{equation}
\paragraph{Moment of Intertia} Design MOI against Euler's bucking by:
\begin{equation}
    I \geq \frac{3FL^2}{\pi^2E}
\end{equation}
\paragraph{Radius of Gyration} Design radius of gyration against slenderness ratio by:
\begin{equation}
    r \geq \frac{L}{200}
\end{equation}
\section{Beam}
\paragraph{Navier's equation} 
\begin{equation}
    \sigma = \frac{My}{I}
\end{equation}

\paragraph{Curvature Equation}
\begin{equation}
    \phi = \frac{M}{EI}
\end{equation}

\paragraph{MAT 1} The change in slope between two points is given by the first moment area theorem:
\begin{equation}
    \Delta_{AB} = \theta_B - \theta_A = \int_A^B \phi(x) \, dx
\end{equation}

\paragraph{MAT 2} The deviation of point $A$ from the tangent drawn at point $B$ is given by the second moment area theorem:
\begin{equation}
    EIt_{A/B} = \int_B^A x M(x) \, dx = \bar{x}_{AB} \int_B^A M(x) \, dx
\end{equation}
\paragraph{Shear Stress} Given by Jourawski's equation:
\begin{equation}
    \tau = \frac{VQ}{Ib}
\end{equation}
\section{Virtual Work}
\paragraph{Work} In elastic deformation, for internal energy, we have
$$
W_\text{int} = V\frac{\sigma \epsilon}{2} = \frac{P\Delta}{2}
$$
In Hookes Law, for external energy, we have:
$$
W_\text{ext} = F\Delta r
$$
\paragraph{Change of length} For a change of length, we have 
\begin{equation}
\Delta = \frac{PL}{EA}
\end{equation}
\paragraph{Deflection} to calculate bean deflection, we sum the total virtual force multiplied by extension (worked done by victual forces, virtual work):
\begin{equation}
    F^\star \Delta_{\hat{r}} = \sum_{i}P^\star_i \Delta_i
\end{equation}
\section{Vibraion}
\subsection{Free Vibration}
Since for change of length, we have:
\begin{align}
    \Delta = \frac{PL}{EA} \nonumber 
    \intertext{This could be rewritten as: }
    P = \frac{EA}{L}\Delta \nonumber
    \intertext{For stiffness $k$, this could be molded as a simple harmonic motion with:}
    k = \frac{EA}{L}
\end{align}
\paragraph{For Truss} We can use the method of virtual load to determine $\Delta_0$. 
\paragraph{For Beam} We can use the method of MAT to determine $\Delta_0$.
\paragraph{Point Load} The natural frequency is: \begin{equation}
f_n = \frac{15.76}{\sqrt{\Delta_0}}
\end{equation}
\paragraph{Uniform Load} The natural frequency is 
\begin{equation}
f_n = \frac{17.76}{\sqrt{\Delta_0}}
\end{equation}
\paragraph{Dynamic Amplification Factor} Denoted as DAF, for forced vibration at a frequency $f$, DAF is computed as:
\begin{equation}
    \text{DAF} = \frac{1}{\sqrt{(1-(\frac{f}{f_n})^2)^2+ (2\beta \frac{f}{f_n})^2}}
\end{equation}
\paragraph{Amplification} Members would be subject to loads $\text{DAF} \times P$ when the load is vibrating. 
\section{Shear and Local Buckling}
\begin{table}[ht]
\centering
\caption{Summary of plate buckling failure modes}
\begin{tabular}{|l|l|l|}
\hline
\textbf{Failure Mode}                                           & \textbf{Failure Condition}                                                                                      & \textbf{Equation}                          \\ \hline
Buckling of the compressive flange                              & $\sigma = \dfrac{4 \pi^2 E}{12 (1 - \mu^2)} \left( \dfrac{t}{b} \right)^2$                                      & $\sigma = \dfrac{M y}{I}$                                  \\
between the webs                                                &                                                                                                                &                                                           \\ \hline
Buckling of the tips of the                                     & $\sigma = \dfrac{0.425 \pi^2 E }{12 (1 - \mu^2)} \left( \dfrac{t}{b} \right)^2$                                & $\sigma = \dfrac{M y}{I}$                                  \\
compressive flange                                              &                                                                                                                &                                                           \\ \hline
Buckling of the webs due to the                                 & $\sigma = \dfrac{6 \pi^2 E}{12 (1 - \mu^2)} \left( \dfrac{t}{b} \right)^2$                                      & $\sigma = \dfrac{M y}{I}$                                  \\
flexural stresses                                               &                                                                                                                &                                                           \\ \hline
Shear buckling of the webs                                      & $\tau = \dfrac{5 \pi^2 E}{12 (1 - \mu^2)} \left( \left( \dfrac{t}{h} \right)^2 + \left( \dfrac{t}{a} \right)^2 \right)$ & $\tau = \dfrac{V Q}{I b}$                                  \\ \hline
\end{tabular}
\end{table}
\section{Concrete}
\subsection{Material Properties}
\paragraph{Tensile Strength} The compressive strength and the tensile strength of concrete is related as follows:
\begin{equation}
    f'_t = 0.33\sqrt{f'_c}
\end{equation}
\paragraph{Young's Modulus} The compressive strength and the Young's modulus of concrete is related as follows:
\begin{equation}
    E_c = 4730\sqrt{f'_c}
\end{equation}
\begin{shaded}
\textit{Typical Values}:\\
Steel's Young's modulus is usually $E_s = 200,000$ MPa and the yield strength is $f_y = 400$ MPa.
\end{shaded}


The modular ratio $n$ is given as:
\begin{equation}
    n = \frac{E_s}{E_c}
\end{equation}

The quantity of longitudinal reinforcement $\rho$ is given as:
\begin{equation}
    \rho = \frac{A_s}{bd}
\end{equation}
where:
\begin{equation*}
\begin{split}
    A_s = \text{The area of the steel reinforcements.}\\
    b = \text{The width of the cross-sectional region of interest.}\\
    d = \text{The distance from the edge of the region of interest to the opposing reinforcements.}
\end{split}
\end{equation*}  

\paragraph{$k$, Scaling Factor} of extreme compression fiber to the neutral axis is given as:
\begin{equation}
    k = \sqrt{(n\rho)^2 + 2n\rho} - n\rho
\end{equation}
\paragraph{$j$, Scaling factor } of the flexural lever is given as:
\begin{equation}
    j = 1 - \frac{1}{3}k
\end{equation}
\begin{shaded}
    \textit{Typical Values:}\\  The scaler factors are usually $k = \frac{3}{8}$ and $j = \frac{7}{8}$.
\end{shaded}

\subsection{Flexural Stress Analysis}
\paragraph{In Reinforcement} The stress is given by:
\begin{equation}
    f_s = \frac{M}{A_s j d}
\end{equation}
\paragraph{In Concrete} The stress is given by:
\begin{equation}
    f_c = \frac{k}{1-k} \cdot \frac{M}{n A_s j d}
\end{equation}

\subsection{Shear Stress Analysis}
\paragraph{Maximum Shear Stress in Cracked Concrete}
The maximum shear stress $v$ in a cracked concrete member's web is given by:
\begin{equation}
    v = \frac{V}{b_w j d}
\end{equation}
where $b_w$ is the effective web width.

\paragraph{Buckling Shear Stress}
The shear stress $v_{\text{max}}$ that causes buckling from diagonal compression is:
\begin{equation}
    v_{\text{max}} = 0.25 f'_c
\end{equation}
\paragraph{Steps for checking Shear Stress} Steps by step, if the shear strength does not pass the conditions below:
\begin{itemize}
    \item \textbf{Concrete Crushing Limit} Concrete will crush when:
    \begin{equation}
        V \geq \min(V_r, V_{\text{max}})
    \end{equation}
    where:
    \begin{equation}
    V_{\text{max}} = 0.25 f'_c b_w j d
    \end{equation}
    \item \textbf{Shear Strength of the Member}
    The shear strength $V_r$ of the member is:
    \begin{equation}
        V_r = V_c + V_s
    \end{equation}
    \item \textbf{Safety Factor for Design}
    For design purposes, select $V_r$ such that:
    \begin{equation}
        V_r = 0.5 V_c + 0.5 V_s \leq 0.5 V_{\text{max}}
    \end{equation}
\end{itemize}
Then we pass on another case  below if it involves $V_r$
\begin{itemize}
    \item \textbf{Without Reinforcement} If no shear reinforcement is present, the shear strength $V_c$ of the concrete is:
    \begin{equation}
        V_c = 230 \sqrt{f'_c} + 0.9 d b_w j d
    \end{equation}
    \item \textbf{With Minimum Reinforcement} When using shear reinforcement (stirrups), the shear strength $V_c$ of the concrete is:
    \begin{equation}
        V_c = 0.18 \sqrt{f'_c} b_w j d
    \end{equation}
    This equation is valid if:
    \begin{align}
        \frac{A_v f_y}{b_w s} 
 & \geq 0.06 \sqrt{f'_c}\\
        \Leftrightarrow s & \leq \frac{A_v f_y}{0.06 \cdot b_w \cdot \sqrt{f'_c}} \nonumber
    \end{align}
    where:
    \begin{equation*}
    \begin{split}
    A_v= \text{The effective area of stirrups; and}\\
    s = \text{Spacing between stirrups}
    \end{split}
    \end{equation*}
    \item \textbf{With Additional Reinforcement} If shear reinforcement is used, the maximum shear force $V_s$ carried is:
    \begin{equation}
        V_s = \frac{A_v f_y j d}{s} \cot(35^\circ)
    \end{equation}
\end{itemize}

\paragraph{Design Suggestions for Safety}
If a design is unsafe, consider the following:
\begin{itemize}
    \item If $V \geq 0.5 V_{\text{max}}$, resize the cross-section.
    \item If $V \geq 0.5 V_c$, add reinforcements.
    \item If $V \geq 0.5 V_c + 0.5 V_s$, adjust the spacing $s$:
    \begin{equation}
    s = \frac{0.5 A_v f_y j d \cot(35^\circ)}{V - 0.5 \times 0.18 \sqrt{f'_c} b_w j d}
    \end{equation}
\end{itemize}
\appendix
\newpage
\rhead{Calc II Appendix}
\lhead{\leftmark}
\pagenumbering{alph}
\section{Trigonometric Derivatives}
$$
\setlength{\tabcolsep}{8pt} % Adjust column spacing
\renewcommand{\arraystretch}{1.5} % Adjust row spacing
\begin{array}{|l|l|}
\hline
\textbf{Trigonometric Function} & \textbf{Inverse Trigonometric Function} \\ \hline
\dfrac{d}{dx} \text{sin}(x) = \text{cos}(x) & \dfrac{d}{dx} \text{arcsin}(x) = \dfrac{1}{\sqrt{1 - x^2}} \\ \hline
\dfrac{d}{dx} \text{cos}(x) = -\text{sin}(x) & \dfrac{d}{dx} \text{arccos}(x) = \dfrac{-1}{\sqrt{1 - x^2}} \\ \hline
\dfrac{d}{dx} \text{tan}(x) = \text{sec}^2(x) & \dfrac{d}{dx} \text{arctan}(x) = \dfrac{1}{1 + x^2} \\ \hline
\dfrac{d}{dx} \text{cot}(x) = -\text{csc}^2(x) & \dfrac{d}{dx} \text{arccot}(x) = \dfrac{-1}{1 + x^2} \\ \hline
\dfrac{d}{dx} \text{sec}(x) = \text{sec}(x)\text{tan}(x) & \dfrac{d}{dx} \text{arcsec}(x) = \dfrac{1}{|x|\sqrt{x^2 - 1}} \\ \hline
\dfrac{d}{dx} \text{csc}(x) = -\text{csc}(x)\text{cot}(x) & \dfrac{d}{dx} \text{arccsc}(x) = \dfrac{-1}{|x|\sqrt{x^2 - 1}} \\ \hline
\hline
\textbf{Hyperbolic Function} & \textbf{Inverse Hyperbolic Function} \\ \hline
\dfrac{d}{dx} \text{sinh}(x) = \text{cosh}(x) & \dfrac{d}{dx} \text{arsinh}(x) = \dfrac{1}{\sqrt{x^2 + 1}} \\ \hline
\dfrac{d}{dx} \text{cosh}(x) = \text{sinh}(x) & \dfrac{d}{dx} \text{arcosh}(x) = \dfrac{1}{\sqrt{x^2 - 1}} \quad (x > 1) \\ \hline
\dfrac{d}{dx} \text{tanh}(x) = \text{sech}^2(x) & \dfrac{d}{dx} \text{artanh}(x) = \dfrac{1}{1 - x^2} \quad (|x| < 1) \\ \hline
\dfrac{d}{dx} \text{coth}(x) = -\text{csch}^2(x) & \dfrac{d}{dx} \text{arcoth}(x) = \dfrac{1}{1 - x^2} \quad (|x| > 1) \\ \hline
\dfrac{d}{dx} \text{sech}(x) = -\text{sech}(x)\text{tanh}(x) & \dfrac{d}{dx} \text{arsech}(x) = \dfrac{-1}{x\sqrt{1 - x^2}} \quad (0 < x \leq 1) \\ \hline
\dfrac{d}{dx} \text{csch}(x) = -\text{csch}(x)\text{coth}(x) & \dfrac{d}{dx} \text{arcsch}(x) = \dfrac{-1}{|x|\sqrt{x^2 + 1}} \\ \hline
\end{array}
$$
\newpage
\section{Polar Curves}\label{app:polarCurves}
\paragraph{} Below are some common polar curves:
\begin{figure}[h!]
    \centering
    \includegraphics[width=0.9\textwidth]{AppendixItems/polarCurves.png}
    \caption{Polar Curves}
    \label{fig:polarCurves}
\end{figure}
Click: \ref{sec:polar} to go back to the section on polar coordinates.
\newpage
\section{Trigonometric Identities}\label{app:trigIdentities}
\begin{table}
    \centering
    \begin{tabular}{|c|c|}
        \hline
        \textbf{Identity} & \textbf{Value} \\
        \hline
        $\sin^2\theta + \cos^2\theta = 1$ & $1$ \\
        $\tan^2\theta = \sec^2\theta$ - 1 & $\sec^2\theta - 1$ \\
        $\tan\theta = \frac{\sin\theta}{\cos\theta}$ & $\frac{\sin\theta}{\cos\theta}$ \\
        $\cot\theta = \frac{\cos\theta}{\sin\theta}$ & $\frac{\cos\theta}{\sin\theta}$ \\
        $\sec\theta = \frac{1}{\cos\theta}$ & $\frac{1}{\cos\theta}$ \\
        $\csc\theta = \frac{1}{\sin\theta}$ & $\frac{1}{\sin\theta}$ \\
        $\sin(-\theta) = -\sin\theta$ & $-\sin\theta$ \\
        $\cos(-\theta) = \cos\theta$ & $\cos\theta$ \\
        $\tan(-\theta) = -\tan\theta$ & $-\tan\theta$ \\
        $\cot(-\theta) = -\cot\theta$ & $-\cot\theta$ \\
        $\sec(-\theta) = \sec\theta$ & $\sec\theta$ \\
        $\csc(-\theta) = -\csc\theta$ & $-\csc\theta$ \\
        $\sin(\theta \pm \phi) = \sin\theta\cos\phi \pm \cos\theta\sin\phi$ & $\sin\theta\cos\phi \pm \cos\theta\sin\phi$ \\
        $\cos(\theta \pm \phi) = \cos\theta\cos\phi \mp \sin\theta\sin\phi$ & $\cos\theta\cos\phi \mp \sin\theta\sin\phi$ \\
        $\tan(\theta \pm \phi) = \frac{\tan\theta \pm \tan\phi}{1 \mp \tan\theta\tan\phi}$ & $\frac{\tan\theta \pm \tan\phi}{1 \mp \tan\theta\tan\phi}$ \\
        $\sin(2\theta) = 2\sin\theta\cos\theta$ & $2\sin\theta\cos\theta$ \\
        $\cos(2\theta) = \cos^2\theta - \sin^2\theta$ & $\cos^2\theta - \sin^2\theta$ \\
        $2\cos(2\theta) = 2\cos^2\theta - 1 = 1 - 2\sin^2\theta$ & $2\cos^2\theta - 1 = 1 - 2\sin^2\theta$ \\
        $\tan(2\theta) = \frac{2\tan\theta}{1 - \tan^2\theta}$ & $\frac{2\tan\theta}{1 - \tan^2\theta}$ \\
        \hline
    \end{tabular}
    \caption{Trigonometric Identities}
    \label{tab:trigIdentities}
\end{table}
\end{document}