\documentclass[11pt]{article}
\usepackage{mathtools}
\usepackage{amsmath}
\usepackage{amssymb}
\usepackage{amsfonts}
\usepackage{amsthm}
\usepackage{xcolor}
\usepackage{graphicx}
\usepackage[top=2.0cm,bottom=2.0cm,left=2.5cm,right=2.5cm]{geometry}
\usepackage{tikz}
\usepackage{float}
\usepackage{multicol}
\usepackage{pgfplots}
\usepackage{lastpage}
\usepackage{siunitx}
\usepackage{xspace}
\usepackage[labelfont=bf]{caption}
\usepackage[hidelinks, urlcolor=blue, linkcolor=blue, colorlinks=true]{hyperref}
\usepackage[capitalize,noabbrev]{cleveref}
\usepackage[absolute]{textpos}
\usepackage{systeme}

\newcommand{\R}{\mathbb{R}}
\newcommand{\C}{\mathbb{C}}
%% define course title
\newcommand{\course}{MAT185}
\newcommand{\assignmenttitle}{Assignment 1}

%% header and footer
\firstpageheader{}{}{\textbf{{\color{red} Due:} 10:00pm, Tuesday Jan. 21, 2025}}
\firstpageheadrule
\runningheader{}{Page~\thepage~of~\numpages}{\course~--~\assignmenttitle}
\footer{}{}{}

\setlength\parindent{0pt} % no indentation in document

%% formats exam class
\qformat{\textbf{Question \thequestiontitle:}\hfill} % title of question 
\boxedpoints
\pointpoints{mark}{marks}
\pointsinrightmargin
\hpword{Marks:}
\hsword{Your score:}
\unframedsolutions
\totalformat{\boxed{\textnormal{\totalpoints~\if\totalpoints1 mark\else marks\fi}}}
\definecolor{SolutionColor}{rgb}{0,0,1}
\renewcommand{\solutiontitle}{}
\AtBeginEnvironment{solution}{\color{blue}}

% %correct choices in solution
\CorrectChoiceEmphasis{\rm}
\checkedchar{\tikz\draw[blue,fill=blue] (0,0) circle (1ex);}

% % increase distance between checkbox items
\renewcommand{\checkboxeshook}{\setlength{\itemsep}{6pt}}

%% distance between questions and parts
\renewcommand{\questionshook}{\setlength{\parsep}{10pt}}
\renewcommand{\partshook}{\setlength{\parsep}{15pt}}

%% define arrows in text
\newcommand{\arrow}{$\rightarrow$\xspace}
\newcommand{\Arrow}{$\Rightarrow$\xspace}

% % math notation:
%\veccol{1}{2}{3}
\newcommand{\veccol}[3]{
    \begin{bmatrix}
        #1\\
        #2\\
        #3\\
    \end{bmatrix}}
  
%\vecrow{1}{2}{3}
\newcommand{\vecrow}[3]{\left[#1~#2~ #3\right]}

%\matrixTwo{1}{2}{3}{4}
\newcommand{\matrixTwo}[4]{\left[\begin{array}{cc}#1&#2\\#3&#4\end{array}\right]}

% \matrixThree{1}{2}{3}{4}{5}{6}{7}{8}{9}
\newcommand{\matrixThree}[9]{\left[\begin{array}{ccc}#1&#2&#3\\#4&#5&#6\\#7&#8&#9\end{array}\right]}

%\matrixCorner{1}{2}{3}{4}
\newcommand{\matrixCorner}[4]{\left[\begin{array}{ccc}#1& \cdots&#2\\ \vdots & \ddots & \vdots\\#3&
      \cdots&#4\end{array}\right]}

% \nR
\newcommand{\nR}{{}^{n}\mathbb{R}}
% \Rn
\newcommand{\Rn}{\mathbb{R}^{n}}
% \nRn
\newcommand{\nRn}{{}^{n}\mathbb{R}^{n}}
% \nRm
\newcommand{\nRm}{{}^{n}\mathbb{R}^{m}}
% \nRm
\newcommand{\mRn}{{}^{m}\mathbb{R}^{n}}
% \mRm
\newcommand{\mRm}{{}^{m}\mathbb{R}^{m}}        

% \u
\renewcommand{\u}{{\bf u}}      
% \v
\renewcommand{\v}{{\bf v}}      
% \w
\newcommand{\w}{{\bf w}}    
% \V
\newcommand{\V}{{\bf V}}                   
       
%% define abbreviations
\newcommand{\row}{\operatorname{row}\,}
\newcommand{\col}{\operatorname{col}\,}
\renewcommand{\dim}{\operatorname{dim}\,}
\renewcommand{\span}{\operatorname{span}\,}
\newcommand{\rank}{\operatorname{rank}\,}
\renewcommand{\ker}{\operatorname{ker}\,}
\newcommand{\nul}{\operatorname{null}\,}
\renewcommand{\det}{\operatorname{det}\,}
\newcommand{\adj}{\operatorname{adj}\,}

\usepackage{xcolor}
% Sean's original colours:
%\definecolor{dkrgreen}{rgb}{0.1, 0.4, 0.3} 
\definecolor{dkrgreen}{HTML}{009988} % this is the color-blind friendly teal from below
%\definecolor{dkred}{rgb}{0.8, 0.05, 0.05} 
\definecolor{dkred}{HTML}{EE3377}  % this is the colour-blind friendly magenta from below
%\definecolor{orange}{rgb}{0.8, 0.33, 0.0}
%\definecolor{goldenrod}{rgb}{0.85, 0.65, 0.13}
\definecolor{blue}{HTML}{1965B0} % this is the colour-blind friendly blue from below
%
% colour-blind-friendly colours from https://personal.sron.nl/~pault/
\definecolor{tolBlue}{HTML}{1965B0}
\definecolor{tolMedBlue}{HTML}{5289C7}
\definecolor{tolLightBlue}{HTML}{7BAFDE} 
\definecolor{tolRed}{HTML}{E8601C} 
\definecolor{tolYellow}{HTML}{F6C141}
\definecolor{tolTeal}{HTML}{009988}
%\definecolor{tolBlue}{HTML}{0077BB} 
\definecolor{tolCyan}{HTML}{33BBEE}
\definecolor{tolTeal}{HTML}{009988} 
\definecolor{tolOrange}{HTML}{EE7733} 
%\definecolor{tolRed}{HTML}{CC3311} 
\definecolor{tolMagenta}{HTML}{EE3377} 
\definecolor{tolGrey}{HTML}{BBBBBB}

%%% This command makes a framed box of a chosen height.
\newcommand{\makenonemptybox}[2]{%
\par\nobreak\vspace{\ht\strutbox}\noindent
\setlength{\fboxrule}{0pt} % set this to 0pt to make invisible
\fbox{%
\parbox[c][#1][t]{\dimexpr\linewidth-2\fboxsep}{
  \hrule width \hsize height 0pt
  \vspace{-0.6cm}
  \color{SolutionColor}#2\color{black}
 }%
}%
}


\begin{document}
\thispagestyle{empty}
{\LARGE \bf ECE 159 Lecture Notes}\\
{\large Hei Shing Cheung}\\
Fundamentals of Electric Circuits, Winter 2024 \hfill ECE 159\\
\vspace{10pt}
\section{Basic Concepts}
\subsection{Basic Defintions}
\begin{definition}[Voltage]
    The difference of the electric potential between two points in a circuit.
    \begin{equation}
        V \defeq \int_C \textbf{E} \cdot d\textbf{l}
    \end{equation}
    \end{definition}
    
    \begin{definition}[Current]
    Rate of change of charge at a particular point in the circuit.
    \begin{equation}
        I \defeq \frac{dq}{dt}
    \end{equation}
    \end{definition}
    
    \begin{definition}[Power]
    The rate of change of energy for an element in the circuit. Power is the product of voltage and current.
    \end{definition}
    
    \begin{definition}[Resistors]
    A circuit element that restricts the flow of charge.
    \end{definition}
    
    \begin{definition}[Ideal Sources]
    A circuit element that provides a set voltage or current regardless of what it is connected to.
    \end{definition}
    
    \begin{definition}[Dependent Sources]
    A circuit element that provides a voltage or current to the circuit depending on another voltage or current in the circuit.
    \end{definition}
    
    \begin{definition}[Passive Sign Convention]
    If a positive current flows out of the positive side of a voltage, that element is \textbf{delivering} power. Otherwise, it is \textbf{absorbing} power. All resistors absorb power.
    \end{definition}
\subsection{High School Review}
\paragraph{Electric Field} $\textbf{E}$ is created when charges are separated. It leads to a potential difference.
\paragraph{Worked done by Electric Field} Force due to the field is:
\begin{equation}
    \textbf{F} = q\textbf{E}
\end{equation}
, such that it moves charges and do work.
\paragraph{Capacitor} Energy stored in capacity is:
\begin{equation}
    U_\text{E} = \frac{1}{2}CV^2
\end{equation}
\paragraph{Energy Density} The energy [J/m$^3$] in a dielectric material is:
\begin{equation}
    u_\text{E} = \frac{1}{2}\epsilon_r \epsilon_0 E^2
\end{equation}
where:
\begin{center}
    $\epsilon_0$ = Permittivity of free space \\
    $\epsilon_r$ = Relative Permittivity
\end{center}
\paragraph{Magnetic field} \textbf{B} is created when charges move.
\paragraph{An electromotive force (emf)} can be induced when the magnetic flux
through a loop is changing with time. Induced emf does the same thing as the potential difference of a
battery. 
\paragraph{Induced EMF} An inductor can store energy in a ``magnetic form"
\begin{equation}
    U_B = \frac{1}{2} L i^2 \quad 
\end{equation}
\paragraph{Energy Density}
\begin{equation}
    u_B = \frac{1}{2 \mu_r \mu_0} B^2
\end{equation}
\subsection{Voltage and Current}
\paragraph{} Now instead of dealing with electric and magnetic vector fields, we can
use the scalar quantities of voltage and current:
\begin{align}
    \text{Voltage} &\defeq \text{Difference in Electric Potential}\nonumber \\ 
    \Delta V &= V_A - V_B \nonumber \\
    &= \int_B^A \textbf{E} \cdot dl \nonumber\\
    &= \frac{\Delta W}{q}
\end{align}

\paragraph{Electromotive Force (emf) / Voltage} Work is performed by 
the electric field to move a charge from one point to another. 
The work done per unit charge is the voltage.
\begin{equation}
    \text{Voltage} = \frac{\Delta W}{q}
\end{equation}
\subsection{Power and Energy}
\paragraph{Power} To relate power to  voltage and current, we can use the following equation:
\begin{equation}
    P \defeq \frac{dW}{dt} = \frac{dW}{dq} \cdot \frac{dq}{dt} = V \cdot I
\end{equation}
\paragraph{Sign} If $P = V \cdot I > 0$, the element is delivering power. If $P = V \cdot I < 0$, the element is absorbing power.
\paragraph{Energy}  the energy absorbed or supplied by an element
from time $t_0$ to time $t_1$ is:
\begin{equation}
    W = \int_{t_0}^{t_1} P(t) dt
\end{equation}
\subsection{Circuit Elements}
\paragraph{Active Elements} Active elements are sources of energy. They can deliver power to the circuit. While passive elements are elements that can only absorb power.
\paragraph{Ideal Sources} Ideal sources are sources that provide a set voltage or current regardless of what it is connected to. We often assume that sources are ideal.
\paragraph{Independent Sources} Independent sources are sources that provide a set voltage or current regardless of what it is connected to.
\paragraph{Dependent Sources} Dependent sources are sources that provide a voltage or current to the circuit depending on another voltage or current in the circuit.
\paragraph{Resistor} A resistor is a circuit element that restricts the flow of charge. The voltage across a resistor is proportional to the current through it.

\subsection{Resistance}
\paragraph{Ohm's Law} Ohm's Law states that the current through a resistor is proportional to the voltage across it.
\begin{equation}
    V = IR
\end{equation}
\paragraph{Resistivity} The resistance of a material is proportional to the length of the material and inversely proportional to the cross-sectional area of the material.
\begin{equation}
    R = \rho \frac{l}{A}
\end{equation}
\paragraph{Short and Open Circuits} A short circuit is a circuit with no resistance, while an open circuit is a circuit with infinite resistance.
\paragraph{Conductance} The reciprocal of resistance is conductance. The unit of conductance is mho (ohm spelled backwards), \rotatebox[origin=c]{180}{$\Omega$}.
\begin{equation}
    G = \frac{1}{R}
\end{equation}
\paragraph{Power in Resistors} The power dissipated in a resistor is given by:
\begin{equation}
    P = I^2R = \frac{V^2}{R}
\end{equation}
\paragraph{Resistors in Series} If resistors are placed in series with each other, their current must be the same, The equivalent resistance of resistors in series is the sum of the resistances.
\begin{equation}
    R_{\text{eq}} = R_1 + R_2 + \dots + R_n
\end{equation}
\paragraph{Resistors in Parallel} If resistors are placed in parallel with each other, their voltage must be the same. The equivalent resistance of resistors in parallel is the reciprocal of the sum of the reciprocals of the resistances.
\begin{equation}
    \frac{1}{R_{\text{eq}}} = \frac{1}{R_1} + \frac{1}{R_2} + \dots + \frac{1}{R_n}
\end{equation}
\paragraph{Comination of Series and Parallel Resistors} To find the equivalent resistance of a combination of series and parallel resistors, first find the equivalent resistance of the resistors in series, then find the equivalent resistance of the resistors in parallel.
\subsection{Node, Branch, and Loop}
\paragraph{} Three important ideas in electric circuit analysis are:
\begin{enumerate}
    \item \textbf{Branch}: A branch is an element in an electric circuit, such as a source or a resistor.
    \item \textbf{Node}: A point of connection of two or more branches within an electric circuit.
    \item \textbf{Loop}: Any closed path in an electric circuit. To form a loop, one must start and end at the same point and never pass through the same node twice.
\end{enumerate}
\section{DC Circuits}
\paragraph{}

\end{document}