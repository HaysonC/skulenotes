\documentclass[11pt]{article}
\usepackage{mathtools}
\usepackage{amsmath}
\usepackage{amssymb}
\usepackage{amsfonts}
\usepackage{amsthm}
\usepackage{xcolor}
\usepackage{graphicx}
\usepackage[top=2.0cm,bottom=2.0cm,left=2.5cm,right=2.5cm]{geometry}
\usepackage{tikz}
\usepackage{float}
\usepackage{multicol}
\usepackage{pgfplots}
\usepackage{lastpage}
\usepackage{siunitx}
\usepackage{xspace}
\usepackage[labelfont=bf]{caption}
\usepackage[hidelinks, urlcolor=blue, linkcolor=blue, colorlinks=true]{hyperref}
\usepackage[capitalize,noabbrev]{cleveref}
\usepackage[absolute]{textpos}
\usepackage{systeme}

\newcommand{\R}{\mathbb{R}}
\newcommand{\C}{\mathbb{C}}
%% define course title
\newcommand{\course}{MAT185}
\newcommand{\assignmenttitle}{Assignment 1}

%% header and footer
\firstpageheader{}{}{\textbf{{\color{red} Due:} 10:00pm, Tuesday Jan. 21, 2025}}
\firstpageheadrule
\runningheader{}{Page~\thepage~of~\numpages}{\course~--~\assignmenttitle}
\footer{}{}{}

\setlength\parindent{0pt} % no indentation in document

%% formats exam class
\qformat{\textbf{Question \thequestiontitle:}\hfill} % title of question 
\boxedpoints
\pointpoints{mark}{marks}
\pointsinrightmargin
\hpword{Marks:}
\hsword{Your score:}
\unframedsolutions
\totalformat{\boxed{\textnormal{\totalpoints~\if\totalpoints1 mark\else marks\fi}}}
\definecolor{SolutionColor}{rgb}{0,0,1}
\renewcommand{\solutiontitle}{}
\AtBeginEnvironment{solution}{\color{blue}}

% %correct choices in solution
\CorrectChoiceEmphasis{\rm}
\checkedchar{\tikz\draw[blue,fill=blue] (0,0) circle (1ex);}

% % increase distance between checkbox items
\renewcommand{\checkboxeshook}{\setlength{\itemsep}{6pt}}

%% distance between questions and parts
\renewcommand{\questionshook}{\setlength{\parsep}{10pt}}
\renewcommand{\partshook}{\setlength{\parsep}{15pt}}

%% define arrows in text
\newcommand{\arrow}{$\rightarrow$\xspace}
\newcommand{\Arrow}{$\Rightarrow$\xspace}

% % math notation:
%\veccol{1}{2}{3}
\newcommand{\veccol}[3]{
    \begin{bmatrix}
        #1\\
        #2\\
        #3\\
    \end{bmatrix}}
  
%\vecrow{1}{2}{3}
\newcommand{\vecrow}[3]{\left[#1~#2~ #3\right]}

%\matrixTwo{1}{2}{3}{4}
\newcommand{\matrixTwo}[4]{\left[\begin{array}{cc}#1&#2\\#3&#4\end{array}\right]}

% \matrixThree{1}{2}{3}{4}{5}{6}{7}{8}{9}
\newcommand{\matrixThree}[9]{\left[\begin{array}{ccc}#1&#2&#3\\#4&#5&#6\\#7&#8&#9\end{array}\right]}

%\matrixCorner{1}{2}{3}{4}
\newcommand{\matrixCorner}[4]{\left[\begin{array}{ccc}#1& \cdots&#2\\ \vdots & \ddots & \vdots\\#3&
      \cdots&#4\end{array}\right]}

% \nR
\newcommand{\nR}{{}^{n}\mathbb{R}}
% \Rn
\newcommand{\Rn}{\mathbb{R}^{n}}
% \nRn
\newcommand{\nRn}{{}^{n}\mathbb{R}^{n}}
% \nRm
\newcommand{\nRm}{{}^{n}\mathbb{R}^{m}}
% \nRm
\newcommand{\mRn}{{}^{m}\mathbb{R}^{n}}
% \mRm
\newcommand{\mRm}{{}^{m}\mathbb{R}^{m}}        

% \u
\renewcommand{\u}{{\bf u}}      
% \v
\renewcommand{\v}{{\bf v}}      
% \w
\newcommand{\w}{{\bf w}}    
% \V
\newcommand{\V}{{\bf V}}                   
       
%% define abbreviations
\newcommand{\row}{\operatorname{row}\,}
\newcommand{\col}{\operatorname{col}\,}
\renewcommand{\dim}{\operatorname{dim}\,}
\renewcommand{\span}{\operatorname{span}\,}
\newcommand{\rank}{\operatorname{rank}\,}
\renewcommand{\ker}{\operatorname{ker}\,}
\newcommand{\nul}{\operatorname{null}\,}
\renewcommand{\det}{\operatorname{det}\,}
\newcommand{\adj}{\operatorname{adj}\,}

\usepackage{xcolor}
% Sean's original colours:
%\definecolor{dkrgreen}{rgb}{0.1, 0.4, 0.3} 
\definecolor{dkrgreen}{HTML}{009988} % this is the color-blind friendly teal from below
%\definecolor{dkred}{rgb}{0.8, 0.05, 0.05} 
\definecolor{dkred}{HTML}{EE3377}  % this is the colour-blind friendly magenta from below
%\definecolor{orange}{rgb}{0.8, 0.33, 0.0}
%\definecolor{goldenrod}{rgb}{0.85, 0.65, 0.13}
\definecolor{blue}{HTML}{1965B0} % this is the colour-blind friendly blue from below
%
% colour-blind-friendly colours from https://personal.sron.nl/~pault/
\definecolor{tolBlue}{HTML}{1965B0}
\definecolor{tolMedBlue}{HTML}{5289C7}
\definecolor{tolLightBlue}{HTML}{7BAFDE} 
\definecolor{tolRed}{HTML}{E8601C} 
\definecolor{tolYellow}{HTML}{F6C141}
\definecolor{tolTeal}{HTML}{009988}
%\definecolor{tolBlue}{HTML}{0077BB} 
\definecolor{tolCyan}{HTML}{33BBEE}
\definecolor{tolTeal}{HTML}{009988} 
\definecolor{tolOrange}{HTML}{EE7733} 
%\definecolor{tolRed}{HTML}{CC3311} 
\definecolor{tolMagenta}{HTML}{EE3377} 
\definecolor{tolGrey}{HTML}{BBBBBB}

%%% This command makes a framed box of a chosen height.
\newcommand{\makenonemptybox}[2]{%
\par\nobreak\vspace{\ht\strutbox}\noindent
\setlength{\fboxrule}{0pt} % set this to 0pt to make invisible
\fbox{%
\parbox[c][#1][t]{\dimexpr\linewidth-2\fboxsep}{
  \hrule width \hsize height 0pt
  \vspace{-0.6cm}
  \color{SolutionColor}#2\color{black}
 }%
}%
}


\begin{document}
\thispagestyle{empty}
{\LARGE \bf ESC 195 Lecture Notes}\\
{\large Hei Shing Cheung}\\
Caculus II, Winter 2024 \hfill ESC 195\\
\section{More on Integrals}
\subsection{Riemann Sum - Non-Uniform Petition}
\begin{example}
    Given the following definite integral:
    $$\int^2_0 \sqrt{x} dx$$, we cannot evaluate its Riemann sum with uniforms partition, since the series of root cannot be easily evaluated. 
\end{example}

The definite integral of $\sqrt{x}$ from $0$ to $2$ using a Riemann sum with a non-uniform partition is given by:

$$
\int_{0}^{2} \sqrt{x} \, dx = \lim_{n \to \infty} \sum_{i=1}^n \sqrt{x_i} \Delta x_i
$$

where:
\begin{itemize}
    \item $x_0 = 0, \, x_n = 2$,
    \item $x_i = i^2 \cdot \frac{2}{n^2}$ for $i = 0, 1, 2, \dots, n$,
    \item $\Delta x_i = x_i - x_{i-1} = \frac{2}{n^2} \cdot (2i - 1)$.
\end{itemize}

The Riemann sum becomes:
$$
S_n = \sum_{i=1}^n \sqrt{i^2 \cdot \frac{2}{n^2}} \cdot \frac{2}{n^2} \cdot (2i - 1).
$$

Simplifying further:
$$
S_n = \sum_{i=1}^n \sqrt{\frac{2i^2}{n^2}} \cdot \frac{2}{n^2} \cdot (2i - 1).
$$

Taking the limit as $n \to \infty$, the sum converges to the exact value of the integral using the series of squares. 
$$
\int_{0}^{2} \sqrt{x} \, dx = \frac{4\sqrt{2}}{3}.
$$
\paragraph{Condition} $n \to \infty$ Ensures $\Delta x_i \to 0$
\paragraph{} As $n \to \infty$, the partition points $x_i$ become increasingly dense. This ensures that the partition becomes infinitely fine.

\subsection{Integration By Parts}
Using the product rule:
$$
\frac{d}{dx} \big[ u(x) v(x) \big] = u'(x)v(x) + u(x)v'(x),
$$
integrating both sides with respect to $x$ gives:
$$
u(x)v(x) = \int u'(x) v(x) \, dx + \int u(x) v'(x) \, dx.
$$
Rearranging this:
$$
\int u'(x) v(x) \, dx = u(x)v(x) - \int u(x) v'(x) \, dx.
$$

\paragraph{Integration by parts formula}
\begin{equation}
\int u \, dv = uv - \int v \, du.
\end{equation}
\begin{example}
We want to solve the integral $$ \int x e^{2x} \, dx $$ using integration by parts.

Let:
$$ u = x, \quad dv = e^{2x} \, dx. $$

Then, we compute the derivatives and integrals:
$$ du = dx, \quad v = \frac{e^{2x}}{2}. $$

Now, apply the integration by parts formula:
$$ \int u \, dv = uv - \int v \, du. $$

Substituting in the values:
$$ \int x e^{2x} \, dx = x \cdot \frac{e^{2x}}{2} - \int \frac{e^{2x}}{2} \, dx. $$

Next, compute the remaining integral:
$$ \int \frac{e^{2x}}{2} \, dx = \frac{e^{2x}}{4}. $$

Thus, the result is:
$$ \int x e^{2x} \, dx = \frac{x e^{2x}}{2} - \frac{e^{2x}}{4} + C. $$

\end{example}

\begin{example}
We want to solve $$ \int x^2 \sin(2x) \, dx $$ using double integration by parts.

First, let:
$$ u = x^2, \quad dv = \sin(2x) \, dx. $$

Then:
$$ du = 2x \, dx, \quad v = -\frac{1}{2} \cos(2x). $$

Using the IBP formula:
$$ \int u \, dv = uv - \int v \, du, $$

we get:
$$ \int x^2 \sin(2x) \, dx = -\frac{x^2}{2} \cos(2x) + \int x \cos(2x) \, dx. $$

Now, apply IBP again to \( \int x \cos(2x) \, dx \), let:
$$ u = x, \quad dv = \cos(2x) \, dx. $$

Then:
$$ du = dx, \quad v = \frac{1}{2} \sin(2x). $$

Using the IBP formula again:
$$ \int x \cos(2x) \, dx = \frac{x}{2} \sin(2x) - \int \frac{1}{2} \sin(2x) \, dx, $$

and solving the remaining integral:
$$ \int \frac{1}{2} \sin(2x) \, dx = -\frac{1}{4} \cos(2x). $$

Thus, the final result is:
$$ \int x^2 \sin(2x) \, dx = -\frac{x^2}{2} \cos(2x) + \frac{x}{2} \sin(2x) + \frac{1}{4} \cos(2x) + C. $$

\end{example}
\paragraph{} 
\end{document}
