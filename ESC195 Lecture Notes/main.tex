\documentclass[11pt]{article}
\usepackage[utf8]{inputenc}	% Para caracteres en español
\usepackage{amsmath,amsthm,amsfonts,amssymb,amscd}
\usepackage{multirow,booktabs}
\usepackage[table]{xcolor}
\usepackage{fullpage}
\usepackage{lastpage}
\usepackage{enumitem}
\usepackage{fancyhdr}
\usepackage{mathrsfs}
\usepackage{wrapfig}
\usepackage{setspace}
\usepackage{hyperref}
\usepackage{calc}
\usepackage{multicol}
\usepackage{cancel}
\usepackage[retainorgcmds]{IEEEtrantools}
\usepackage[margin=3cm]{geometry}
\usepackage{amsmath}
\newlength{\tabcont}
\setlength{\parindent}{0.0in}
\setlength{\parskip}{0.05in}
\usepackage{empheq}
\usepackage{framed}
\usepackage[most]{tcolorbox}
\usepackage{xcolor}
\colorlet{shadecolor}{orange!15}
\parindent 0in
\parskip 12pt
\geometry{margin=1in, headsep=0.25in}
\theoremstyle{definition}
\usepackage{pdfpages}
\newtheorem{defn}{Definition}
\newtheorem{reg}{Rule}
\newtheorem{exer}{Exercise}
\newtheorem{note}{Note}
\usepackage{fancyhdr}\usepackage{xcolor}\usepackage{amsmath}\usepackage{amssymb}\pagestyle{fancy}\rhead{}
\newtheorem{theorem}{Theorem}[subsection]
\theoremstyle{definition}
\newtheorem{definition}[theorem]{Definiton}
\newtheorem{example}[theorem]{Example}
\newtheorem{corollary}[theorem]{Corollary}
\newtheorem{lemma}[theorem]{Lemma}
\title{Chapter 9 Review Notes}
\begin{document}
\thispagestyle{empty}
{\LARGE \bf ESC 195 Lecture Notes}\\
{\large Hei Shing Cheung}\\
Caculus II, Winter 2024 \hfill ESC 195\\
\section{More on Integrals}
\subsection{Riemann Sum - Non-Uniform Petition}
\begin{example}
    Given the following definite integral:
    $$\int^2_0 \sqrt{x} dx$$, we cannot evaluate its Riemann sum with uniforms partition, since the series of root cannot be easily evaluated. 
\end{example}

The definite integral of $\sqrt{x}$ from $0$ to $2$ using a Riemann sum with a non-uniform partition is given by:

$$
\int_{0}^{2} \sqrt{x} \, dx = \lim_{n \to \infty} \sum_{i=1}^n \sqrt{x_i} \Delta x_i
$$

where:
\begin{itemize}
    \item $x_0 = 0, \, x_n = 2$,
    \item $x_i = i^2 \cdot \frac{2}{n^2}$ for $i = 0, 1, 2, \dots, n$,
    \item $\Delta x_i = x_i - x_{i-1} = \frac{2}{n^2} \cdot (2i - 1)$.
\end{itemize}

The Riemann sum becomes:
$$
S_n = \sum_{i=1}^n \sqrt{i^2 \cdot \frac{2}{n^2}} \cdot \frac{2}{n^2} \cdot (2i - 1).
$$

Simplifying further:
$$
S_n = \sum_{i=1}^n \sqrt{\frac{2i^2}{n^2}} \cdot \frac{2}{n^2} \cdot (2i - 1).
$$

Taking the limit as $n \to \infty$, the sum converges to the exact value of the integral using the series of squares. 
$$
\int_{0}^{2} \sqrt{x} \, dx = \frac{4\sqrt{2}}{3}.
$$
\paragraph{Condition} $n \to \infty$ Ensures $\Delta x_i \to 0$
\paragraph{} As $n \to \infty$, the partition points $x_i$ become increasingly dense. This ensures that the partition becomes infinitely fine.

\subsection{Integration By Parts}
Using the product rule:
$$
\frac{d}{dx} \big[ u(x) v(x) \big] = u'(x)v(x) + u(x)v'(x),
$$
integrating both sides with respect to $x$ gives:
$$
u(x)v(x) = \int u'(x) v(x) \, dx + \int u(x) v'(x) \, dx.
$$
Rearranging this:
$$
\int u'(x) v(x) \, dx = u(x)v(x) - \int u(x) v'(x) \, dx.
$$

\paragraph{Integration by parts formula}
\begin{equation}
\int u \, dv = uv - \int v \, du.
\end{equation}
\begin{example}
We want to solve the integral $$ \int x e^{2x} \, dx $$ using integration by parts.

Let:
$$ u = x, \quad dv = e^{2x} \, dx. $$

Then, we compute the derivatives and integrals:
$$ du = dx, \quad v = \frac{e^{2x}}{2}. $$

Now, apply the integration by parts formula:
$$ \int u \, dv = uv - \int v \, du. $$

Substituting in the values:
$$ \int x e^{2x} \, dx = x \cdot \frac{e^{2x}}{2} - \int \frac{e^{2x}}{2} \, dx. $$

Next, compute the remaining integral:
$$ \int \frac{e^{2x}}{2} \, dx = \frac{e^{2x}}{4}. $$

Thus, the result is:
$$ \int x e^{2x} \, dx = \frac{x e^{2x}}{2} - \frac{e^{2x}}{4} + C. $$

\end{example}

\begin{example}
We want to solve $$ \int x^2 \sin(2x) \, dx $$ using double integration by parts.

First, let:
$$ u = x^2, \quad dv = \sin(2x) \, dx. $$

Then:
$$ du = 2x \, dx, \quad v = -\frac{1}{2} \cos(2x). $$

Using the IBP formula:
$$ \int u \, dv = uv - \int v \, du, $$

we get:
$$ \int x^2 \sin(2x) \, dx = -\frac{x^2}{2} \cos(2x) + \int x \cos(2x) \, dx. $$

Now, apply IBP again to \( \int x \cos(2x) \, dx \), let:
$$ u = x, \quad dv = \cos(2x) \, dx. $$

Then:
$$ du = dx, \quad v = \frac{1}{2} \sin(2x). $$

Using the IBP formula again:
$$ \int x \cos(2x) \, dx = \frac{x}{2} \sin(2x) - \int \frac{1}{2} \sin(2x) \, dx, $$

and solving the remaining integral:
$$ \int \frac{1}{2} \sin(2x) \, dx = -\frac{1}{4} \cos(2x). $$

Thus, the final result is:
$$ \int x^2 \sin(2x) \, dx = -\frac{x^2}{2} \cos(2x) + \frac{x}{2} \sin(2x) + \frac{1}{4} \cos(2x) + C. $$

\end{example}
\paragraph{} 
\end{document}
