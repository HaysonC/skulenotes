\documentclass[10pt,addpoints,answers]{exam}
\usepackage{mathtools}
\usepackage{amsmath}
\usepackage{amssymb}
\usepackage{amsfonts}
\usepackage{amsthm}
\usepackage{xcolor}
\usepackage{graphicx}
\usepackage[top=2.0cm,bottom=2.0cm,left=2.5cm,right=2.5cm]{geometry}
\usepackage{tikz}
\usepackage{float}
\usepackage{multicol}
\usepackage{pgfplots}
\usepackage{lastpage}
\usepackage{siunitx}
\usepackage{xspace}
\usepackage[labelfont=bf]{caption}
\usepackage[hidelinks, urlcolor=blue, linkcolor=blue, colorlinks=true]{hyperref}
\usepackage[capitalize,noabbrev]{cleveref}
\usepackage[absolute]{textpos}
\usepackage{systeme}

\newcommand{\R}{\mathbb{R}}
\newcommand{\C}{\mathbb{C}}
%% define course title
\newcommand{\course}{MAT185}
\newcommand{\assignmenttitle}{Assignment 1}

%% header and footer
\firstpageheader{}{}{\textbf{{\color{red} Due:} 10:00pm, Tuesday Jan. 21, 2025}}
\firstpageheadrule
\runningheader{}{Page~\thepage~of~\numpages}{\course~--~\assignmenttitle}
\footer{}{}{}

\setlength\parindent{0pt} % no indentation in document

%% formats exam class
\qformat{\textbf{Question \thequestiontitle:}\hfill} % title of question 
\boxedpoints
\pointpoints{mark}{marks}
\pointsinrightmargin
\hpword{Marks:}
\hsword{Your score:}
\unframedsolutions
\totalformat{\boxed{\textnormal{\totalpoints~\if\totalpoints1 mark\else marks\fi}}}
\definecolor{SolutionColor}{rgb}{0,0,1}
\renewcommand{\solutiontitle}{}
\AtBeginEnvironment{solution}{\color{blue}}

% %correct choices in solution
\CorrectChoiceEmphasis{\rm}
\checkedchar{\tikz\draw[blue,fill=blue] (0,0) circle (1ex);}

% % increase distance between checkbox items
\renewcommand{\checkboxeshook}{\setlength{\itemsep}{6pt}}

%% distance between questions and parts
\renewcommand{\questionshook}{\setlength{\parsep}{10pt}}
\renewcommand{\partshook}{\setlength{\parsep}{15pt}}

%% define arrows in text
\newcommand{\arrow}{$\rightarrow$\xspace}
\newcommand{\Arrow}{$\Rightarrow$\xspace}

% % math notation:
%\veccol{1}{2}{3}
\newcommand{\veccol}[3]{
    \begin{bmatrix}
        #1\\
        #2\\
        #3\\
    \end{bmatrix}}
  
%\vecrow{1}{2}{3}
\newcommand{\vecrow}[3]{\left[#1~#2~ #3\right]}

%\matrixTwo{1}{2}{3}{4}
\newcommand{\matrixTwo}[4]{\left[\begin{array}{cc}#1&#2\\#3&#4\end{array}\right]}

% \matrixThree{1}{2}{3}{4}{5}{6}{7}{8}{9}
\newcommand{\matrixThree}[9]{\left[\begin{array}{ccc}#1&#2&#3\\#4&#5&#6\\#7&#8&#9\end{array}\right]}

%\matrixCorner{1}{2}{3}{4}
\newcommand{\matrixCorner}[4]{\left[\begin{array}{ccc}#1& \cdots&#2\\ \vdots & \ddots & \vdots\\#3&
      \cdots&#4\end{array}\right]}

% \nR
\newcommand{\nR}{{}^{n}\mathbb{R}}
% \Rn
\newcommand{\Rn}{\mathbb{R}^{n}}
% \nRn
\newcommand{\nRn}{{}^{n}\mathbb{R}^{n}}
% \nRm
\newcommand{\nRm}{{}^{n}\mathbb{R}^{m}}
% \nRm
\newcommand{\mRn}{{}^{m}\mathbb{R}^{n}}
% \mRm
\newcommand{\mRm}{{}^{m}\mathbb{R}^{m}}        

% \u
\renewcommand{\u}{{\bf u}}      
% \v
\renewcommand{\v}{{\bf v}}      
% \w
\newcommand{\w}{{\bf w}}    
% \V
\newcommand{\V}{{\bf V}}                   
       
%% define abbreviations
\newcommand{\row}{\operatorname{row}\,}
\newcommand{\col}{\operatorname{col}\,}
\renewcommand{\dim}{\operatorname{dim}\,}
\renewcommand{\span}{\operatorname{span}\,}
\newcommand{\rank}{\operatorname{rank}\,}
\renewcommand{\ker}{\operatorname{ker}\,}
\newcommand{\nul}{\operatorname{null}\,}
\renewcommand{\det}{\operatorname{det}\,}
\newcommand{\adj}{\operatorname{adj}\,}

\usepackage{xcolor}
% Sean's original colours:
%\definecolor{dkrgreen}{rgb}{0.1, 0.4, 0.3} 
\definecolor{dkrgreen}{HTML}{009988} % this is the color-blind friendly teal from below
%\definecolor{dkred}{rgb}{0.8, 0.05, 0.05} 
\definecolor{dkred}{HTML}{EE3377}  % this is the colour-blind friendly magenta from below
%\definecolor{orange}{rgb}{0.8, 0.33, 0.0}
%\definecolor{goldenrod}{rgb}{0.85, 0.65, 0.13}
\definecolor{blue}{HTML}{1965B0} % this is the colour-blind friendly blue from below
%
% colour-blind-friendly colours from https://personal.sron.nl/~pault/
\definecolor{tolBlue}{HTML}{1965B0}
\definecolor{tolMedBlue}{HTML}{5289C7}
\definecolor{tolLightBlue}{HTML}{7BAFDE} 
\definecolor{tolRed}{HTML}{E8601C} 
\definecolor{tolYellow}{HTML}{F6C141}
\definecolor{tolTeal}{HTML}{009988}
%\definecolor{tolBlue}{HTML}{0077BB} 
\definecolor{tolCyan}{HTML}{33BBEE}
\definecolor{tolTeal}{HTML}{009988} 
\definecolor{tolOrange}{HTML}{EE7733} 
%\definecolor{tolRed}{HTML}{CC3311} 
\definecolor{tolMagenta}{HTML}{EE3377} 
\definecolor{tolGrey}{HTML}{BBBBBB}

%%% This command makes a framed box of a chosen height.
\newcommand{\makenonemptybox}[2]{%
\par\nobreak\vspace{\ht\strutbox}\noindent
\setlength{\fboxrule}{0pt} % set this to 0pt to make invisible
\fbox{%
\parbox[c][#1][t]{\dimexpr\linewidth-2\fboxsep}{
  \hrule width \hsize height 0pt
  \vspace{-0.6cm}
  \color{SolutionColor}#2\color{black}
 }%
}%
}



\begin{document}

\vspace*{-0.5cm}
\begin{center}
  \large
  \textbf{\Large \course~--~Linear Algebra}\\[0.1cm]
  \textbf{\assignmenttitle}
\end{center}
\bigskip

\textbf{\large Instructions:}\\
\normalsize

Please read the {\bf MAT185 Assignment Policies \& FAQ} document for details on
submission policies, collaboration rules and academic integrity, and general
instructions.

\begin{enumerate}


\item \textbf{Submissions are only accepted by}
  \href{https://www.gradescope.ca}{Gradescope}. Do not send anything by email.
  Late submissions are not accepted under any circumstance. Remember you can
  resubmit anytime before the deadline.

\item \textbf{Submit solutions using only this template pdf}.  Your submission
  should be a single pdf with your full written solutions for each question. If
  your solution is not written using this template pdf (scanned print or
  digital) then your submission will not be assessed. Organize your work neatly
  in the space provided.  Do not submit rough work.

\item \textbf{Show your work and justify your steps} on every question but do
  not include extraneous information.  Put your final answer in the box
  provided, if necessary.  We recommend you write draft solutions on separate
  pages and afterwards write your polished solutions here on this template.

\item \textbf{You must fill out and sign the academic integrity statement
    below}; otherwise, you will receive zero for this assignment.

\end{enumerate}

\vspace{10pt}


\textbf{\large Academic Integrity Statement:} \\

%%% Student information

% Student 1
\fbox{
  \begin{minipage}{\textwidth}
    \vspace{0.75cm}
    \makebox[\textwidth]{\large Full Name: Cheung, Hei Shing\enspace\hrulefill}\\[0.75cm]
    \makebox[\textwidth]{\large Student number: 1010907823\enspace\hrulefill}\\
  \end{minipage}
}

\vspace*{0.1in}

% Student 2
\fbox{
  \begin{minipage}{\textwidth}
    \vspace{0.75cm}
    \makebox[\textwidth]{\large Full Name: Lin, Zhiyue\enspace\hrulefill}\\[0.75cm]
    \makebox[\textwidth]{\large Student number: 1010900759\enspace\hrulefill}\\
  \end{minipage}
}

\bigskip
\large \textbf{I confirm that:}
\normalsize

\begin{itemize} 
\item I have read and followed the policies described in the document {\bf
    MAT185 Assignment Policies \& FAQ}.
\item In particular, I have read and understand the rules for
  collaboration, and permitted resources on assignments as described in
  subsection II of the the aforementioned document. I have not violated
  these rules while completing and writing this assignment.
\item I have not used generative AI in writing this assignment.
\item I understand the consequences of violating the University's academic
  integrity policies as outlined in the
  \href{http://www.governingcouncil.utoronto.ca/policies/behaveac.htm}{Code of
    Behaviour on Academic Matters}. I have not violated them while completing
  and writing this assignment.
\end{itemize}
\bigskip

\large \textbf{By submitting this assignment to Gradescope, I agree that the
  statements above are true.}  \normalsize

\newpage

\begin{questions}
  \question In this problem, you will prove that the elementary operations you
  learnt in ESC103 do not change the set of solutions of a system of linear
  equations.  Consider the four linear systems
$$
\hspace{.8cm} \mathcal{A}: \hspace{.5cm} \left\{ \arraycolsep=1.4pt
	\begin{array}{rcrrcrcr}
          3x &+&2y &+&2z &= &9\\
          11x &+&7y &+&3z &=& 15\\
          3x &+&2y &+&z &=& 5
	\end{array} \right.
      \hspace{1.9cm}
      \mathcal{B}: \hspace{.5cm} \left\{
        \arraycolsep=1.4pt
	\begin{array}{rcrcrccr}
          3x &+&2y &+&2z &= &9\\
          11x &+&7y &+&3z &=& 15\\
          (3 + 11 \beta) x &+&(2+7 \beta) y &+&(1+3 \beta) z &=& 5 + 15 \beta
	\end{array} \right.
$$
$$
\hspace{-3.3cm} \mathcal{C}: \hspace{.5cm} \left\{ \arraycolsep=1.4pt
	\begin{array}{rcrcrccr}
          3x &+&2y &+&2z &= &9\\
          11 \alpha x &+&7 \alpha y &+&3 \alpha z &=& 15 \alpha\\
          3x &+&2y &+&z &=& 5
	\end{array} \right.
      \hspace{.9cm}
      \mathcal{D}: \hspace{.5cm} \left\{
        \arraycolsep=1.4pt
	\begin{array}{rcrcrccr}
          11x &+&7y &+&3z &=& 15\\
          3x &+&2y &+&2z &= &9\\
          3x &+&2y &+&z &=& 5
	\end{array} \right.
$$
Above, $x$, $y$, and $z$ are all \textbf{variables}.  The parameters $\alpha$
and $\beta$ are real numbers.

Let $A \subseteq \R^3$ be the \textbf{set of solutions} of the linear system
$\mathcal{A}$.  A point $(a,b,c) \in A$ means that
$$
\hspace{.8cm} \mathcal{A}: \hspace{.5cm} \left\{ \arraycolsep=1.4pt
	\begin{array}{rcrcrccrl}
          3a &+&2b &+&2c &= &9  &\hspace{.5cm} \mbox{\textbf{TRUE}}\\
          11a &+&7b &+&3c &=& 15 & \mbox{\textbf{TRUE}}\\
          3a &+&2b &+&c &=& 5 & \mbox{\textbf{TRUE}}
	\end{array} \right.
$$
Because $a$, $b$, and $c$ are all \textbf{real numbers}, equations involving
them are either true (as in $1+1=2$ is true) or false (as in $1+1=3$ is false).
And so, $(a,b,c)$ is in $A$ (is a solution of $\mathcal{A}$) if all three
equations are true when the variables $x$, $y$, and $z$ take on the values $a$,
$b$, and $c$ respectively.  If there are no points $(a,b,c)$ for which the three
equations are all true then $\mathcal{A}$ has no solutions and
$A = \varnothing$.  \textit{Note: this does not contradict the statement that
  $A \subseteq \R^3$ because $\varnothing \subseteq \R^3$.}

\underline{Pro tip:} If you have an equation where the left-hand side and
right-hand side are real numbers (or are numbers in any field) then the equation
$LHS = RHS$ is \textbf{TRUE} if and only if $LHS-RHS = 0$ and is \textbf{FALSE}
if and only if $LHS-RHS \neq 0$.

Finally, there is the question of how you could show that two sets, $A$ and $B$,
are equal.
\begin{itemize}
\item If you want to prove that $A = B$ and $A = \varnothing$ then you need to
  prove that $B = \varnothing$.  %That is, $A$ and $B$ are both the empty set.
\item If you want to prove that $A = B$ and $B = \varnothing$ then you need to
  prove that $A = \varnothing$.  %That is, $A$ and $B$ are both the empty set.
\item If you want to prove that $A = B$ and neither $A$ nor $B$ are nonempty
  then you can do this by proving
  \begin{itemize}
  \item if $a \in A$ then $a \in B$ (this proves $A \subseteq B$) \textbf{and}
  \item if $b \in B$ then $b \in A$ (this proves $B \subseteq A$).
  \end{itemize}
\end{itemize} If you have sets $A$ and $B$ and you don't know whether or not
they are empty or nonempty, you have do all three steps above.

\textbf{For the curious:} The third case is related to another common proof
technique: if you want to prove that two real numbers are equal, $a=b$, you
first use one idea/approach to prove $a \leq b$ and a different idea/approach to
prove $a \geq b$.  This technique is jarring the first few times one sees it.
This approach works for any totally-ordered field; it doesn't work for the
complex numbers, for example.

\vfil
	
\begin{parts}
  \part Let $A$ be the set of solutions of $\mathcal{A}$ and $B$ be the set of
  solutions of $\mathcal{B}$.  Prove that $A = B$.    \textit{Note: you don't know
    whether or not $\mathcal{A}$ or $\mathcal{B}$ have any solutions.  This
    means that you must address all three possible cases: $A = \varnothing$,
    $B = \varnothing$, and neither $A$ nor $B$ are the empty set.}
    
    \textbf{Please start your answer on the next page, not on this one.  The grader will not look at this page.  You can continue your answer on the top of page 4, if needed.}
    
  
  \newpage

   \textcolor{black}{\underline{case 1: $A = \varnothing$}}
  %%% Do not change the height of this box. Your work must fit inside it.
  \makenonemptybox{130pt}{
    \paragraph{Note} I denote the $k^{\text{th}}$ equation of $\mathcal{P}$ as $\mathcal{P}_k$ for linear system $\mathcal{P} \in \{\mathcal{A}, \mathcal{B}, \mathcal{C}, \mathcal{D}\}$.


    %%% Your work goes here!  
   \paragraph{} We prove that $B = \varnothing$. Assume, for the sake of contradiction, that $B \neq \varnothing$. Then there exists a point $(a,b,c) \in B$ such that
       $\mathcal{B}_1$, $\mathcal{B}_2$, and $\mathcal{B}_3$ are \textbf{TRUE}. Thus:
       \begin{itemize}
        \item For all $\beta \in \mathbb{R}$, $\beta\mathcal{B}_2: 11\beta a + 7\beta b + 3\beta c = 15\beta$ is \textbf{TRUE}; also
        \item $\mathcal{B}_3$ is \textbf{TRUE}. 
       \end{itemize}
    Hence, $\mathcal{B}_3 - \beta \mathcal{B}_2$ is \textbf{TRUE} for all $\beta \in \mathbb{R}$. This implies there exist $(a,b,c)$ s.t.: $3a + 2b + c = 5$ is \textbf{TRUE}. It follows from $\mathcal{A}_3: 3a + 2b + c = 5$ and that $\mathcal{B}_2 = \mathcal{A}_2$ and that $\mathcal{B}_1 = \mathcal{A}_1$ are all \textbf{TRUE}. Hence $(a,b,c) \in A$. Thus, $A \neq \varnothing$, a contradiction. Therefore, $B = \varnothing$.



% end of your answer to case 1 unless you need to continue the answer on page 4
}



   \textcolor{black}{ \underline{case 2: $B = \varnothing$} }
   %%% Do not change the height of this box. Your work must fit inside it.
  \makenonemptybox{130pt}{
   %%% Your work goes here!  
   \paragraph{} We prove that $A = \varnothing$. Assume, for the sake of contradiction, that $A \neq \varnothing$. Then there exists a point $(a,b,c) \in A$ such that $\mathcal{A}_1$, $\mathcal{A}_2$, and $\mathcal{A}_3$ are \textbf{TRUE}. Thus:
       \begin{itemize}
        \item For all $\beta \in \mathbb{R}$, $\beta\mathcal{A}_2: 11\beta a + 7\beta b + 3\beta c = 15\beta$ is \textbf{TRUE}; also
        \item $\mathcal{A}_3$ is \textbf{TRUE}. 
       \end{itemize}
    Hence, $\mathcal{B}_3 = \mathcal{A}_3 + \beta \mathcal{A}_2$ is \textbf{TRUE} for all $\beta \in \mathbb{R}$. It follows from $\mathcal{A}_2 = \mathcal{B}_2$ and that $\mathcal{A}_1 = \mathcal{B}_1$ are all \textbf{TRUE}. Hence $(a,b,c) \in B$. Thus, $B \neq \varnothing$, a contradiction. Therefore, $A = \varnothing$.
% end of your answer to case 2 unless you need to continue the answer on page 4
}


   \textcolor{black}{ \underline{case 3: both $A$ and $B$ are nonempty} }
   %%% Do not change the height of this box. Your work must fit inside it.
  \makenonemptybox{120pt}{
   %%% Your work goes here!  
   \paragraph{} ($A \subseteq B$) Take $(a_1, a_2, a_3) \in A$. Then $\mathcal{A}_1$, $\mathcal{A}_2$, and $\mathcal{A}_3$ are \textbf{TRUE}. Thus:
       \begin{itemize}
        \item For all $\beta \in \mathbb{R}$, $\beta\mathcal{A}_2: 11\beta a_1 + 7\beta a_2 + 3\beta a_3 = 15\beta$ is \textbf{TRUE}; also
        \item $\mathcal{A}_3$ is \textbf{TRUE}. 
       \end{itemize}
    Hence, $\mathcal{B}_3 = \mathcal{A}_3 + \beta \mathcal{A}_2$ is \textbf{TRUE} for all $\beta \in \mathbb{R}$. It follows from $\mathcal{A}_2 = \mathcal{B}_2$ and that $\mathcal{A}_1 = \mathcal{B}_1$ are all \textbf{TRUE}. Thus, $(a_1, a_2, a_3) \in B$. Therefore, $A \subseteq B$.

    \paragraph{} ($B \subseteq A$) Take $(b_1, b_2, b_3) \in B$. Then $\mathcal{B}_1$, $\mathcal{B}_2$, and $\mathcal{B}_3$ are \textbf{TRUE}. Thus:
       \begin{itemize}
        \item For all $\beta \in \mathbb{R}$, $\beta\mathcal{B}_2: 11\beta b_1 + 7\beta b_2 + 3\beta b_3 = 15\beta$ is \textbf{TRUE}; also
        \item $\mathcal{B}_3$ is \textbf{TRUE}. 
       \end{itemize}
    Hence, $\mathcal{A}_3 = \mathcal{B}_3 - \beta \mathcal{B}_2$ is \textbf{TRUE} for all $\beta \in \mathbb{R}$. It follows from $\mathcal{B}_2 = \mathcal{A}_2$ and that $\mathcal{B}_1 = \mathcal{A}_1$ are all \textbf{TRUE}. Thus, $(b_1, b_2, b_3) \in A$. Therefore, $B \subseteq A$.


% end of your answer to case 3 unless you need to continue the answer on page 4
}


  \newpage
  
  \textbf{You can continue your answer here.  If you do so, please make it clear which of the
  three cases you're continuing!}

    %%% Do not change the height of this box. Your work must fit inside it.
  \makenonemptybox{232pt}{
    %%% Your work goes here!  Here is where you can put the continuation of your work from page 3
    %%% to page 4.  It is your responsibility to trim down your answer so that it doesn't run into 
    %%% the next question
    
    
    %%% end of your answer    
  }
 
 %\vspace{1cm}
 
 \part Let $A$ be the set of solutions of $\mathcal{A}$ and $C$ be the set of
solutions of $\mathcal{C}$.
 
 \begin{subparts} \subpart The system $\mathcal{C}$ has a parameter $\alpha \in
\R$.  Under what conditions on $\alpha$ is $A \subseteq C$?  \textit{No justification needed,
just one complete sentence.}
 
\vspace{.15in}

   %%% Do not change the height of this box. Your work must fit inside it.
  \makenonemptybox{25pt}{ 
    %%% Your work goes here!  It is your responsibility to trim down your answer so that it doesn't 
    $\alpha \in \mathbb{R}$. It can be any real number.
    %%% end of your answer    
  }
 
  \vspace{.15in}
  
  \subpart Under what conditions on $\alpha$ is $C \subseteq A$? \textit{No justification needed,
just one complete sentence.} \vspace{.15in}
\makenonemptybox{25pt}{ 
    %%% Your work goes here!  It is your responsibility to trim down your answer so that it doesn't 
    $\alpha \neq 0$. It can be any non zero real number.
    %%% end of your answer    
  }
 
  \subpart Assuming the condition of part ii. holds, prove that $C \subseteq A$.
\textit{You may assume that $C \neq \varnothing$; we'll assume that if you can
address the $C = \varnothing$ case based on your methods in part (a).}\\

%%% Do not change the height of this box. Your work must fit inside it.
  \makenonemptybox{25pt}{
    %%% Your work goes here!  It is your responsibility to trim down your answer so that it doesn't 
    \paragraph{} Given nonempty $A,\, C$, take $(a, b, c) \in C$ such that $\mathcal{C}_1$, $\mathcal{C}_2$, and $\mathcal{C}_3$ are \textbf{TRUE}.  For all $\alpha \neq 0$, we have $\frac{1}{\alpha} \in \mathbb{R}$. So $\frac{1}{\alpha}\mathcal{C}_2$ is \textbf{TRUE}. Thus $\mathcal{B}_2 = \mathcal{C}_2$ is \textbf{TRUE}. It follows that $\mathcal{C}_3 = \mathcal{B}_3$ is \textbf{TRUE} and that $\mathcal{C}_1 = \mathcal{B}_1$ is \textbf{TRUE}. Therefore, $(a, b, c) \in A$. Hence, $C \subseteq A$. 



    %%% end of your answer      
  }

 \end{subparts}

 \newpage
 
 \begin{tikzpicture}
\path (0,0) (.5,0) (.5,.5) (0,.5) (0,0);
\end{tikzpicture}
 
 \part Let $D$ be the set of solutions of $\mathcal{D}$.  Prove that $A = D$.
\textit{You may assume that both $A$ and $D$ are nonempty; we'll assume you can
address the case where either of them is the empty set based on your work in
part (a).}  

\textbf{This part is worth zero points.  Your proof will not be
graded.  If you got stuck, or wrote a proof
you aren't quite sure of, please post to piazza or come to office hours.}

\textbf{YOU MUST UPLOAD THIS PAGE EVEN IF YOU WROTE NOTHING ON IT.}


 
%%% Do not change the height of this box. Your work must fit inside it.

\makenonemptybox{500pt}{
    %%% Your work goes here!  It is your responsibility to trim down your answer so that it doesn't 
    \paragraph{} We have $\mathcal{A}_1 = \mathcal{D}_2$ and $\mathcal{A}_2 = \mathcal{D}_1$ and $\mathcal{A}_3 = \mathcal{D}_3$. Given nonempty $A,\, B$, we have:

    \paragraph{} ($A \subseteq D$) Take $(a_1, a_2, a_3) \in A$. Then $\mathcal{A}_1$, $\mathcal{A}_2$, and $\mathcal{A}_3$ are \textbf{TRUE}. Thus, in respective order:
       \begin{itemize}
        \item $\mathcal{D}_2$ is \textbf{TRUE}; also
        \item $\mathcal{D}_1$ is \textbf{TRUE}; also
        \item $\mathcal{D}_3$ is \textbf{TRUE},
       \end{itemize} 
       so, it follows that $(a_1, a_2, a_3) \in D$. Therefore, $A \subseteq D$.


    \paragraph{} ($D \subseteq A$) Take $(d_1, d_2, d_3) \in D$. Then $\mathcal{D}_1$, $\mathcal{D}_2$, and $\mathcal{D}_3$ are \textbf{TRUE}. Thus, in respective order:
        \begin{itemize}
          \item $\mathcal{A}_2$ is \textbf{TRUE}; also
          \item $\mathcal{A}_1$ is \textbf{TRUE}; also
          \item $\mathcal{A}_3$ is \textbf{TRUE},
        \end{itemize}
        so, it follows that $(d_1, d_2, d_3) \in A$. Therefore, $D \subseteq A$.

    %%% end of your answer
 }

 \newpage
 
 \part At this point, you should have proven that the three elementary
operations did not change the set of solutions of the linear system
$\mathcal{A}$.  The key insights of your proofs didn't rely on $\mathcal{A}$'s
having exactly three equations and three variables and having coefficients $3$,
$2$, $11$ and so forth.  We asked you to think about a simple, concrete example
so that you could focus on the elementary operations and not be distracted by
notation.  Now we are asking you to face the notation and prove something about
a general linear system.
 
 Consider a system of $m$ linear equations with $n$ variables
 $$
 \mathcal{A}: \hspace{.5cm} \left\{ \sum_{j=1}^n a_{ij} x_j = b_i \qquad 1 \leq
i \leq m \right.
 $$
 where the coefficients $a_{ij}$ and constants $b_i$ are all real numbers.  Let
$\mathcal{B}$ be the linear system where the $i_0$th equation of $\mathcal{A}$
has been multiplied by $\alpha \neq 0$.

 Assume that both $\mathcal{A}$ and $\mathcal{B}$ have solutions.  Prove that
$\mathcal{A}$ and $\mathcal{B}$ have the same set of solutions.  

\textbf{This part is worth zero points.  Your proof will not be
graded.  If you got stuck, or wrote a proof
you aren't quite sure of, please post to piazza or come to office hours.}

\textbf{YOU MUST UPLOAD THIS PAGE EVEN IF YOU WROTE NOTHING ON IT.}


%%% Do not change the height of this box. Your work must fit inside it.
\makenonemptybox{200pt}{
    %%% Your work goes here!  It is your responsibility to trim down your answer so that it doesn't 
    \paragraph{} We have $\mathcal{A}_i = \mathcal{B}_i$ for all $i \neq i_0$ and $\mathcal{B}_{i_0} = \alpha \mathcal{A}_{i_0}$. Given nonempty $A,\, B$, we have:
    \begin{equation*}
      \mathcal{A}_{i_0} \text{ is \textbf{TRUE}} \implies \alpha \mathcal{A}_{i_0} \text{ is \textbf{TRUE}} \implies \mathcal{B}_{i_0} \text{ is \textbf{TRUE}}.
    \end{equation*}
    Therefore, $\mathcal{A}$ and $\mathcal{B}$ have the same set of solutions.

    %%% end of your answer   
 }

 
 \newpage
 
 \part Consider writing down the augmented matrix of a linear system
$\mathcal{A}$ and applying elementary row operations to the matrix until you
have a matrix in reduced row echelon form (RREF).  Let $\mathcal{R}$ be a linear
system represented by that RREF matrix.  Explain why $\mathcal{A}$ and
$\mathcal{R}$ have the same set of solutions.  \textit{There's a lot of white
space here but this isn't to indicate that you need to write at length to answer
this question.  You should be able to write your explanation w/ 100 or fewer
words (in the space above the horizontal line). }

%%% Do not change the height of this box. Your work must fit inside it.
\makenonemptybox{250pt}{
    %%% Your work goes here!  It is your responsibility to trim down your answer so that it doesn't 
    Consider the augmented matrix of $\mathcal{A}$ and apply elementary row operations to obtain the RREF matrix of $\mathcal{A}$. For each elementary row operation, the set of solutions of $\mathcal{A}$ remains unchanged. Thus, $\mathcal{A}$ and $\mathcal{R}$ have the same set of solutions.
    %%% end of your answer  
}

 \noindent\rule{.93\textwidth}{1pt}
 
\newpage
 
 \end{parts}
 
 
 \question One of the goals of this question is to get you into good shape for
True/False questions on exams.
 
 For questions (b), (c), and (d) below, please assume the ``natural'' vector
addition and scalar multiplication.  For example, for functions assume the
addition and multiplication given in section 4.2 of Medici.  The ones given in
the ``An Unusual Vector Space'' example are completely valid but are considered
``unnatural'' for this problem.
 
 \begin{parts}
 \part Prove the following lemma by showing that all the vector space axioms
hold.  It's very useful and once you've proven it you can use it whenever you
want.
 
\noindent{\bf \color{dkrgreen} Lemma}: Let $\V$ be a vector space over the
real numbers. If $\V_0 \subseteq \V$ contains ${\bf 0}$ and,
with the same vector addition and scalar multiplication of $\V$,
 is closed under vector
addition and scalar multiplication 
then $\V_0$ is a vector space over the
real numbers.

%%% Do not change the height of this box. Your work must fit inside it.
\makenonemptybox{400pt}{
    %%% Your work goes here!  It is your responsibility to trim down your answer so that it doesn't 
    \paragraph{Additve Axioms}
    \begin{enumerate}
      \item By assumption, $\V_0$ is closed under vector addition. Thus, for all $\mathbf{u}, \mathbf{v} \in \V_0$, $\mathbf{u} + \mathbf{v} \in \V_0$.
      \item By assumption, $\V_0$ has the same vector addition as $\V$ and $\V$ is a vetor space. Thus, for all $\mathbf{u}, \mathbf{v}, \mathbf{w} \in \V_0$, $(\mathbf{u} + \mathbf{v}) + \mathbf{w} = \mathbf{u} + (\mathbf{v} + \mathbf{w})$.
      \item By assumption, $\V_0$ contains $\mathbf{0}$ and $\V$ is a vector space. Thus, for all $\mathbf{u} \in \V_0 \subseteq \V$, $\mathbf{u} + \mathbf{0} = \mathbf{u}$.
      \item For all $\textbf{u} \in \V_0 \subseteq \V$, a vector space. Under the usual scalar multiplication, $0\textbf{u} = \mathbf{0}$. With $(1 + (-1)) = 0$, we have:
      \begin{equation*}
        \begin{aligned}
          1\textbf{u} + (-1)\textbf{u} &= (1 + (-1))\textbf{u} \\
          &= 0\textbf{u} \\
          &= \mathbf{0}.
        \end{aligned}
      \end{equation*}
      Since $\V_0$ is closed under scalar multiplication, $(-1)\textbf{u} \in \V_0$. Thus, $\V_0$ contains the additive inverse of $\textbf{u}$.
      \end{enumerate}
    \paragraph{Scalar Multiplication Axioms}
    \begin{enumerate}
      \item By assumption, $\V_0$ is closed under scalar multiplication. Thus, for all $c \in \mathbb{R}$ and $\textbf{u} \in \V_0$, $c\textbf{u} \in \V_0$.
      \item By assumption, $\V_0$ has the same scalar multiplication as $\V$ and $\V$ is a vector space. Thus, for all $c, d \in \mathbb{R}$ and $\textbf{u} \in \V_0 \subseteq \V$, $(cd)\textbf{u} = c(d\textbf{u})$.
      \item By assumption, $\V_0$ has the same scalar multiplication as $\V$ and $\V$ is a vector space. Thus, for all $c \in \mathbb{R}$ and $\textbf{u}, \textbf{v} \in \V_0 \subseteq \V$, $c(\textbf{u} + \textbf{v}) = c\textbf{u} + c\textbf{v}$.
      \item By assumption, $\V_0$ has the same scalar multiplication as $\V$ and $\V$ is a vector space. Thus, for all $\textbf{u} \in \V_0 \subseteq \V$, $1\textbf{u} = \textbf{u}$.
    \end{enumerate}
      

    %%% end of your answer     
}

 \newpage

 \begin{EnvUplevel} \textbf{True or False:} Unsupported answers will not receive
full credit. Organize your work in a reasonably neat and coherent way.\\[0.25cm]
Indicate your final answers by \textbf{filling in exactly one circle} for each
part below (unfilled \tikz\draw[black] (0,0) circle (.8ex); \, filled
\tikz\draw[black,fill=black] (0,0) circle (.8ex);).

\textbf{Hint: if you think any of the following are vector spaces, ask yourself whether 
there's a way of 
proving that this is true without having to start at the definition and proving that all eight
axioms hold.}

 \end{EnvUplevel}
    
 \part The set $\V$ of all nonpositive real-valued functions on $[2,3]$ 
  with the usual addition and scalar multiplication
 is a
vector space over $\R$.  \textit{Note: a real-valued function on $[2,3]$ is a function
$f: [2,3] \to \R$.  That is, its domain is $[2,3]$ and its range is a subset of
$\R$.  Its range could be all of $\R$, of course.  A function is nonpositive if
$f(x) \leq 0$ for all $x$ in its domain.}
 
 % mark option as correct by using \CorrectChoice
\begin{checkboxes}
  \choice True
  \CorrectChoice False
\end{checkboxes}

 %%% Do not change the height of this box. Your work must fit inside it.
 \makenonemptybox{150pt}{
    %%% Your work goes here!  It is your responsibility to trim down your answer so that it doesn't 
    Pick $f \in \V$ such that there exist $x_0 \in [2,3]$, $f(x_0) \neq 0$. Choose real number $c<0$, then $cf(x_0) > 0$, hence $c\cdot f \notin \V$. Thus, $\V$ is not closed under scalar multiplication. Therefore, $\V$ is not a vector space over $\mathbb{R}$.


    %%% end of your answer   
 }
 
\part The set $\V$ of all real polynomials of degree exactly $n$ 
 with the usual addition and scalar multiplication
is a vector
space.  \textit{Note: A ``real polynomial'' is a polynomial whose coefficients
are all real numbers.}

% mark option as correct by using \CorrectChoice
\begin{checkboxes}
  \choice True
  \CorrectChoice False
\end{checkboxes}

 %%% Do not change the height of this box. Your work must fit inside it.
 \makenonemptybox{100pt}{
    %%% Your work goes here!  It is your responsibility to trim down your answer so that it doesn't 
    Pick $p, q \in \V$ for which $A, -A \in \mathbb{R}$ such that $p(x) = Ax^n$ and $q(x) = -Ax^n$. Then $p(x) + q(x) = A_nx^n + (-A_n)x^n = 0x^n = 0$. However, 0 is not a polynomial of degree exactly $n$. Thus, $\V$ is not closed under vector addition. Therefore, $\V$ is not a vector space over $\mathbb{R}$.

    %%% end of your answer    
 }

\newpage
 \begin{EnvUplevel} \textbf{True or False:} Unsupported answers will not receive
full credit. Organize your work in a reasonably neat and coherent way.\\[0.25cm]
Indicate your final answers by \textbf{filling in exactly one circle} for each
part below (unfilled \tikz\draw[black] (0,0) circle (.8ex); \, filled
\tikz\draw[black,fill=black] (0,0) circle (.8ex);).

\textbf{Hint: if you think the following is a vector space, ask yourself whether 
there's a way of 
proving that this is true without having to start at the definition and proving that all eight
axioms hold.}
 \end{EnvUplevel}


\part The set of $2 \times 2$ upper triangular real matrices with
the usual entry-wise addition and scalar multiplication
is a vector
space over the real numbers.  

\textit{Notes: An $n \times n$ matrix ${\bf A} = [a_{ij}]$ is upper triangular if
$a_{ij}=0$ for all $1 \leq j < i \leq n$.  A matrix is real if all of its entries are
real numbers.  All integer matrices and all rational matrices are also real
matrices.  After doing this problem, ask yourself how your answer would apply
to $n \times n$ upper triangular matrices.}


% mark option as correct by using \CorrectChoice
\begin{checkboxes}
  \CorrectChoice True
  \choice False
\end{checkboxes}

 %%% Do not change the height of this box. Your work must fit inside it.
 \makenonemptybox{175pt}{
    %%% Your work goes here!  It is your responsibility to trim down your answer so that it doesn't 
    Given that $^2\mathbb{R}^2$ is a vector space, the set of $2 \times 2$ upper triangular real matrices is a subset of $^2\mathbb{R}^2$. Thus, we prove that $\V$ is a vector space of $^2\mathbb{R}^2$ by showing subspace test holds.
    \begin{enumerate}
      \item $\mathbf{0} \in \V$: $\mathbf{0}$ is an upper triangular matrix with $\textbf{0} = \begin{bmatrix}
        0 & 0 \\
        0 & 0
      \end{bmatrix}$. Given usual entry-wise addition, for all $\textbf{A} \in \V$, $\textbf{A} + \textbf{0} = \begin{bmatrix}
        a_{11} + 0 & a_{12} + 0 \\
        0 & a_{22} + 0
      \end{bmatrix} = \textbf{A}$. Thus, $\mathbf{0} \in \V$.
      \item Closure under addition: Let $A, B \in \V$. Then $A + B = \begin{bmatrix}
        a_{11} + b_{11} & a_{12} + b_{12} \\
        0 & a_{22} + b_{22}
      \end{bmatrix}$. So $A + B$ is upper triangular, and $A + B \in \V$.
      \item Closure under scalar multiplication: Let $A \in \V$ and $c \in \mathbb{R}$. Then $cA = \begin{bmatrix}
        ca_{11} & ca_{12} \\
        0 & ca_{22}
      \end{bmatrix}$. So $cA$ is upper triangular, and $cA \in \V$.
    \end{enumerate}



    %%% end of your answer     
 }

 
 \end{parts}
\end{questions}

\end{document}
